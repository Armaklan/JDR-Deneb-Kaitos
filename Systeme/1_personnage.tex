\chapter{Le Personnage}

\begin{multicols}{2}

\section{Le Concept}

Le concept correspond à une description rapide du personnage, un premier jet. Que fais-il ? Quel est sa profession ? Qu'est-ce qui le mptive ? A quoi ressemble t'il ? Avoir une bonne idée de son personnage est nécessaire pour effectuer la création. La création de personnage ci-dessous permettra de préciser davantages le concept, et de lui attribuer un profil technique.

\section{Héroïsme et Tension}

\subsection{Le Niveau d'Héroïsme}

Le niveau d'Héroïsme est un paramètre du système Fusina permettant de faire varier légèrement l'ambiance ressentie. Il permet de passer d'un système où les personnages sont à peine au dessus du commun, à un système où ils sont de grands et terribles héros.

\begin{itemize}
\item Un niveau d'héroïsme 0 correspond à des personnages dont rien ne différencie du commun des mortels. 
\item Un niveau d'héroïsme de 1 correspond à des personnages juste au dessus de la moyenne. 
\item Un niveau d'héroïsme de 2 correspond à des personnages héroïques capable d'actions spectaculaires. Les personnages sont toutefois loin d'être invincibles et tomberont bien souvent sur plus puissant qu'eux. 
\item Un niveau d'Héroïsme de 3 correspond à des héros ayant de la bouteille, des vétérans capable d'influencer réellement sur leur univers et connus de tous.
\item Au delà, je ne répond plus de rien !
\end{itemize}

Le niveau d'héroïsme va influer lors de la création, mais également durant le jeu.

Dans l'univers de Deneb Kaitos les personnages débutants ont un niveau d'héroïsme de 2. Le maitre du jeu est biensur libre d'augmenter ce score pour obtenir des personnages plus puissants.

\subsection{Le Niveau de Tension}

Le niveau de Tension reflète la capacité de dépassement des personnages, la tension dramatique. Plus le niveau de Tension est élevé, plus les personnages pourront aller au-delà de leurs limites et faire ponctuellement des actions spectaculaires, mais plus le retour de bâton sera également brutal.

\begin{itemize}
\item Un niveau de tension 0 entraine des résultats plus prévisibles, un monde sans grande surprise.
\item Un niveau de tension 1 indique un univers où de temps en temps survint un coup de chance.
\item Un niveau de tension 2 indique un univers ou les surprises sont assez régulières, mais ou parfois la providence semble s'acharner sur les personnages.
\end{itemize}

Pour la plupart des univers le niveau de tension 1 est conseillé. Le niveau de tension 2 est utile quand vous désirez donner aux personnages un peu plus de peps, mais "pas gratuitement". Je déconseille l'utilisation d'un niveau de tension plus élevé.

Le niveau de tension n'a aucun impact durant la création. Son effet se fait sentir uniquement durant la partie.

Dans l'univers de Deneb Kaitos, les personnages débutants ont un niveau de tension de 2.

\section{Traits et faiblesse}

\subsection{Les Traits}

Les traits sont des adjectifs ou des phrases qui caractérisent le personnage. Ils peuvent caractériser son apparence, son histoire, son caractère.

Voici quelques exemples de traits :
\begin{itemize}
\item Forte-tête 
\item Muscle d'acier
\item Agilité féline
\item Membre de la prêtrise Teldrim
\item Une femme dans chaque astroport
\item ...
\end{itemize}

Les traits n'ont pas de valeur chiffrée pour les représenter. Nous considérons ici que tous les traits ont une importance égale.

Un personnage est défini dans Fusina par 4 + Niveau d'heroïsme traits qui représentent des particularités du personnage, des choses qui font qu'il sort du lot du commun des mortels. Par défaut à Mousquetaires de Sang, les joueurs ont donc 6 traits à compléter.

\subsection{Les Faiblesses}

Les faiblesses décrivent les points faibles, les phobies, les défauts, les tics gênants du personnage.

Voici quelques exemples de faiblesses :
\begin{itemize}
\item Bas du front !
\item Peur du vide.
\item Incapable de communiquer avec la gente féminine.
\item Exilé.
\item ...
\end{itemize}

En jeu, les faiblesses peuvent pénaliser le personnage ou forcer le personnage à adopter certaines lignes de conduite. A l'instar des traits, les faiblesses ne sont pas chiffrées, elles sont toutes d'importance égale.
Le joueur possède de base une faiblesse. Il peut choisir d'en prendre une seconde. Cette seconde faiblesse lui permet d'obtenir un trait supplémentaire.

\section{Les Caractéristiques}

Les caractéristiques synthétisent les aspects sous forme de cinqs scores :

\begin{itemize}
\item Physique : représente les capacités physiques (force, endurance, agilité) du personnage
\item Intellect : représente les capacités de mémoire et de raisonnement du personnage
\item Social : représente les capacités de communication et le charisme du personnage
\item Ame : représente la force d'âme, la volonté et la chance du personnage
\item Influence : représente le niveau social du personnage, ses moyens financiers, ses contacts, ...
\end{itemize}

Les caractéristiques ont une valeur qui va en général de d4 à d12+3. Elle est évaluée sur cette échelle :

\begin{itemize}
\item d4 : Faible
\item d6 : Standard
\item d8 : Supérieur à la moyenne
\item d10 : Excellent
\item d12 : Spectaculaire
\item d12+1 à d12+3 : Héroïque 
\end{itemize}

A la création, chaque trait du personnage va être associé avec une caractéristique. Le nombre de trait associé à une caractéristique permet de déterminer sa valeur.
 
\begin{itemize}
\item 0 trait : d4
\item 1 trait : d6
\item 2 traits : d8
\item 3 traits : d10
\item 4 traits : d12
\item 5 traits : d12+1
\end{itemize}

Une fois la création terminée, les associations décidées n'ont pas d'incidence dans la partie.

\end{multicols}

\remarque{Échelons}{

On appelle un échelon la différence entre deux dés consécutifs dans l'échelle donnée ci-dessus.

Par exemple, entre d4 et d8 il y a deux échelons, entre d10 et d12 il n'y en a qu'un.

Cette notion va revenir dans la partie parlant des conflits, gardez-la bien en mémoire.
}

\begin{multicols}{2}

\section{Les Compétences}

Les compétences représentent le vécu du personnage, ce qu'il a acquis par apprentissage ou expérience. 

Les compétences ont une valeur qui va de d4 à d12 selon l'échelle suivante : 

\begin{itemize}
\item d4 : Vagues connaissances
\item d6 : Amateur
\item d8 : Professionnel
\item d10 : Expert
\item d12 : Grand Maître de la discipline
\end{itemize}

Il n'existe pas de valeur supérieure à d12 pour les compétences.

Le joueur dispose de 25 points à dépenser pour inscrire des compétences sur sa feuille. Les compétences sont totalement libres. Le niveau de la compétence dépend du nombre de points investis :

\begin{itemize}
\item d4 : 1 points
\item d6 : 3 points
\item d8 : 6 points
\item d10 : 10 points
\item d12 : 15 points
\end{itemize}

\textbf{Exemples de compétences :}

    \begin{itemize}
        \item Comédien
        \item Armes de tirs
        \item Séduire la donzelle
    \end{itemize}

\section{L'Equipement}

L'équipement d'un personnage représente ses possessions, les objets qu'il possède.

Certains équipements peuvent être représentés par une valeur allant de d4 à d12+3. Un équipement représenté par une valeur est considéré comme apportant un bonus pour les actions qui le concernent.

Pour déterminer l'équipement des personnages, nous conseillons de procéder de la manière suivante.

Le joueur constitue ensuite une liste d'équipement qu'il désire avoir et qu'il est logique que son personnage ait.

Pour chaque équipement, le MJ a 3 solutions : 

\begin{itemize}
\item Accepter l'équipement : l'équipement est logique pour le personnage et ne requiert pas d'être particulièrement riche. 
\item Refuser l'équipement : au contraire, l'équipement peut être totalement illogique pour le personnage (équipement que le personnage ne sait pas utiliser), ou coûter excessivement cher.
\item Demander un jet d'Influence : l'influence est utilisée pour simuler les contacts du personnages, ainsi que ses ressources. Le jet d'influence permet de savoir si le personnage a pu mettre la main sur l'équipement en question, et s'il a dû s'endetter pour le faire.
\end{itemize}

Le MJ peut utiliser l'une ou l'autre de ces conditions, à sa propre convenance. Nous incitons toutefois à utiliser le jet d'influence pour tous les cas litigieux.


\textbf{Quelques équipements...}

    \begin{itemize}
        \item Dague (d4)
        \item Matériel d'escalade (d8)
        \item Tenue de camouflage (d6)
        \item Canon (d10)
    \end{itemize}

\subsection{Les armures}

Les armures forment l'un des seuls cas particuliers. En effet nous n'avons pas trouvé de manière satisfaisante de les gérer. Voilà donc la gestion que l'on propose. 

Une armure est représentée par un score d'encaissement allant de 1 (protection légère) à 4 (protection lourde magique ou technologique). Le score d'encaissement indique combien de niveaux de blessures la protection peut encaisser dans une seule et même scène. Ainsi une protection de 1 permet d'annuler un handicap temporaire, de transformer une blessure légère en handicap, une blessure grave en légère, etc… 

Les niveaux d'encaissements peuvent être utilisés séparément ou ensemble. 

A la scène suivante, l'armure retrouve sa pleine capacité. 

Il est à noter que ce système fonctionne également très bien pour les protections inhérentes à des conflits non physiques : protection mentale contre les attaques psychiques, protections magiques, protection sociale, etc…

\end{multicols}

\exemple{Utiliser les niveaux d'encaissement d'une protection}
{
	Par exemple, si le personnage a une armure de protection 2, il peut soit abaisser d'un niveau deux blessures (par exemple changer une grave en légère, puis plus tard dans la scène, une mortelle en grave), soit tout utiliser en une fois (par exemple changer une grave en handicap).
}

\clearpage

\exemple{La création en synthèse}{

\textbf{Heroisme} : 2
\textbf{Tension} : 2

\textbf{Les traits}

6 traits à déterminer.

\textbf{Les faiblesses}

1 faiblesse par défaut.
Le joueur peut prendre une faiblesse pour obtenir un trait complémentaire.

\textbf{Caractéristique}

Chaque trait doit être lié à une caractéristique. La valeur de la caractéristique est déterminé par le tableau suivant :
\begin{itemize}
\item 0 trait : d4
\item 1 trait : d6
\item 2 traits : d8
\item 3 traits : d10
\item 4 traits : d12
\item 5 traits : d12+1
\end{itemize}

\textbf{Compétences}

Le joueur dispose de 25 points à dépenser. Le niveau de la compétence dépend du nombre de points investis :

\begin{itemize}
\item d4 : 1 points
\item d6 : 3 points
\item d8 : 6 points
\item d10 : 10 points
\item d12 : 15 points
\end{itemize}

\textbf{Equipements}

Le joueur prépare une liste d'équipement qu'il désire. Le MJ lui accorde en fonction du personnage et de son statut (Influence).
}
