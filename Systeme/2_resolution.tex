\chapter{La Résolution d'une action}

\begin{multicols}{2}

\section{En résumé}

La résolution d'une action se déroule comme suit : 

\begin{enumerate}
\item Le joueur indique l'action qu'il veut effectuer.
\item Le MJ indique au joueur la caractéristique, la compétence, et le ou les équipements appropriés.
\item Le MJ fixe un facteur de difficulté (action simple) ou effectue le jet pour le PNJ (opposition).
\item Le joueur lance l'intégralité des dés correspondants aux élements choisis. Il choisit un résultat (en général le plus élevé) et l'annonce au MJ. Les dés sont explosifs : un dé qui indique son résultat maximum est relancé et cumulé.
\item Le MJ annonce le résultat de l'action.
\item Le joueur décrit l'action.
\end{enumerate}

\section{En détail}

\subsection{Annonce de l'action}

Durant cette phase le joueur doit annoncer ce qu'il désire réaliser. L'important ici n'est pas de décrire avec précision le "comment" mais uniquement l'objectif de l'action. Le système de jeu va permet d'évaluer la réussite de l'objectif, le "comment" n'est qu'affaire de description et vient donc après le jet.

\subsection{Choix des élements utilisés}

Le maitre du jeu doit ensuite fixer les élements de la fiche de personnage qui vont être utilisées. En général le maitre du jeu procède ainsi :

\begin{enumerate}
\item Il annonce la caractéristique utilisée en fonction de l'action entreprise. S'il hésite entre deux caractéristiques (cas assez rare), il prend alors la caractéristique la plus avantageuse pour le personnage.
\item Il annonce alors le domaine de compétence à utiliser (discrétion, natation, ...). Le joueur lui propose alors la compétence qu'il juge la plus appropriée. Le MJ est alors libre de l'accepter ou de la refuser. Si le joueur pense n'avoir aucune compétence, alors il fera le jet uniquement avec sa caractéristique ou ses équipements.
\item Le joueur propose le ou les équipements qu'il juge appropriés. Encore une fois, le MJ est libre de les accepter, ou de les refuser.
\item Le MJ indique au joueur si une faiblesse ou un handicap intervient. Bien sûr il est de bon ton que le joueur lui rappelle en cas d'oubli...
\end{enumerate}

\subsection{Bonus permanent}

Quand un personnage possède une caractéristique supérieure à D12, le bonus, allant de +1 à +3, s'applique au résultat final du jet.

\subsection{Définir la difficulté}

Le MJ doit définir un seuil de difficulté vis à vis de l'action entreprise par le personnage. Ce seuil de difficulté est "indépendant" du personnage qui l'entreprend. Les contraintes dues à l'environnement (difficulté de concentration, temps limité, ...) ne sont pas à prendre en compte dans la difficulté. Elles seront exprimées autrement (voir chapitre "Les Faiblesses").

L'échelle de difficulté (indicative) que nous conseillons est la suivante :

\begin{itemize}
\item Action enfantine : 1
\item Action facile : 3
\item Action malaisée : 5
\item Action difficile : 8
\item Action héroïque : 11
\item Action quasi impossible : 13 ou plus.
\end{itemize}

\subsection{Réussite et échec des actions}

Tout n'est pas binaire dans la vie. Il y a des nuances dans la réussite ou l'échec d'une action entreprise par quelqu'un.

La plupart des jeux nuancent peu : on réussit, on échoue, ou une version extrême d'un de ces deux cas, appelée "critique". Nous avons jugé qu'il était essentiel de nuancer les résultats. Cette nuance doit se retrouver à la base même du système.

\begin{itemize}
\item Réussite Totale : L'action qu'entreprend le personnage est réussie, sans aucun souci.
\item Réussite Partielle : L'action qu'entreprend le personnage est au choix réussie, mais avec des soucis, ou n'est pas réussie mais le personnage se rapproche de son objectif.
\item Réussite de justesse : L'action qu'entreprend le personnage est réussie, mais avec de graves soucis.
\item Échec Partiel : L'action qu'entreprend le personnage est au choix ratée, mais le personnage ne s'éloigne pas de l'objectif, ou alors elle réussit mais le personnage n'en tirera pas d'avantage pour se rapprocher de son objectif.
\item Échec Total : L'action qu'entreprend le personnage est ratée, de plus il s'éloigne de son objectif ou s'attire des ennuis.
\end{itemize}

Ces seuils sont indicatifs. Avec l'expérience, vous apprendrez à faire peser chaque point, chaque marge différente, dans la balance pour évaluer le résultat d'une action.

\subsection{Détermination du degré de réussite/échec}

En fonction de la différence entre le résultat (obtenu comme détaillé plus haut) et la difficulté, le MJ annonce au joueur le cas dans lequel il se trouve. C'est ensuite à ce dernier de décrire ce qu'il se passe. Le MJ a bien entendu un droit de rectification ou de véto.

On considère encore la réussite ou l'échec en fonction de la différence entre la difficulté et le résultat du jet sur cette échelle :

\begin{itemize}
\item Le résultat est de plus de 3 points supérieur à la difficulté : c'est une réussite totale.
\item Le résultat est entre 1 et 3 points supérieur à la difficulté : c'est une réussite partielle.
\item Le résultat est égal à la difficulté : c'est une réussite de justesse.
\item Le résultat est entre 1 et 3 points inférieur à la difficulté : c'est un échec partiel.
\item Le résultat est de plus de 3 points inférieur à la difficulté : c'est un échec total.
\end{itemize}

\subsection{Description du résultat}

Une fois que le MJ a déterminé la réussite ou l'échec de l'action, c'est au joueur de décrire l'action en détails et son résultat. Le joueur est libre de la décrire et de l'interpréter comme il l'entend. Le MJ peut, toutefois, décider à tout moment de mettre son véto sur un élement particulier, ou de compléter la description.

La description doit bien sûr être en accord avec les résultats obtenus. Si ce n'est pas le cas, il est évident que le MJ doit la réprendre intégralement.


\end{multicols}

\note{Note de conception}{
    Cette idée de laisser les joueurs détailler le résultat de leurs actions provient du jeu "Swashbucklers of the 7 skies", un jeu que nous conseillons particulièrement pour sa mécanique.
}

\exemple{Exemple : Passer au dessus de barbelés}{
    Bob souhaite passer au dessus d'une grille en haut de laquelle se trouve un barbelé, pour entrer dans une maison bien gardée, en plus (et oui, c'est pas sport de découper la grille).
    
    Bref, cette action est incontestablement physique (où Bob a d8). Bob est un mec qui "Grimpe tout ce qui peut se grimper" où il a d10 (Bob aime donner des noms rigolos à ses compétences). La grille fait 2m de haut, avec de l'espace entre les croisillons, l'escalade est facile, c'est de passer les barbelés sans encombre, la difficulté.
    
    Le MJ l'estime donc à 5. Bob lance alors un d8 et un d10. Le d8 donne 7 et le d10 6. Bob choisit donc le plus élevé des dés, donc le résultat de son jet est de 7.
    
    Enfin, on compare le résultat de Bob à la difficulté fixée par le MJ. Il y a une différence de 2, c'est donc une réussite partielle.
}

\exemple{Exemple bis : Le passage au dessus de barbelés}{
    La tentative de Bob est donc une réussite partielle, explique le MJ au joueur. C'est à dire donc que soit l'action est réussie, mais il y a un souci, ou alors elle n'est pas réussie mais quelque chose se passe. Le joueur choisit donc de lui-même (avec l'accord du MJ) ce qui viendra qualifier sa réussite de partielle (par opposition à totale) et non le MJ de manière unilatérale).
    
    Bob explique donc que son pantalon s'accroche aux barbelés du côté extérieur, ce qui le fait basculer la tête en bas. Il ajoute que ses mollets frottent les barbelés, ce qui lui occasionne des tas de petites entailles qui piquent. Il termine en expliquant qu'il perd donc un peu de temps  à se sortir de cette situation.
    
    Il aurait pu choisir de passer sans souci ni blessure, mais en se réceptionnant bruyamment, ce qui n'aurait pas manqué d'alerter les vigiles proches.
    
    Ou encore, il aurait pu expliquer qu'il retombait de l'autre côté de la grille, le barbelé accroché au pantalon. Il aurait alors eu la possibilité de l'enlever sur une petite zone en prenant ses précautions pour ne pas se blesser. Il pourra retenter, mais les barbelés ne seront plus là au moment délicat de passer au dessus de la grille.
}

\begin{multicols}{2}

\section{Handicaps, Faiblesses et Blessures}

Handicaps, Faiblesses, et Blessures sont gérés de manière identique en terme de jeu. La seule différence entre ces trois élements est leur origine et leur durée d'application. Les faiblesses interviennent dans deux cas :

\begin{itemize}
\item Une faiblesse peut inciter un personnage à adopter une ligne de conduite qu'il n'aurait pas adopté normalement. Si le joueur respecte sa faiblesse, il gagnera alors un point de Panache (sans prise en compte du stock limite).
\item Une faiblesse pour également pénaliser un personnage lors d'une action. La première faiblesse qui entre en jeu interdit au joueur de choisir le résultat le plus elevée parmis son jet. Les faiblesses suivantes écartent un dé supplémentaire en commençant toujours par les plus haut. Si tous les dés sont écartés de cette manière, l'action est automatique échouée : e personnage est paralysé par ses faiblesses, trop handicapé, ...
\end{itemize}

\end{multicols}

\exemple{Les faiblesses et leur gestion}{

Bob a beau être un homme courageux, il a particulièrement peur du vide (première faiblesse). Durant sa fuite, il arrive au bord d'une passerelle. Il remarque très vite une poutre permettant de rejoindre la passerelle d'en face. En dessous de la poutre, du vide ! 

En plus de tout cela, quelques minutes avant, Bob a été sérieusement blessé à la jambe (seconde faiblesse).

Bob essaye de passer tranquillement le long de cette poutre. Il a donc son dé de Physique (D8), et sa compétence d'Athlétisme (D6). Étant donné que deux faiblesses entrent en compte, il reste devant la poutre, paralysé.

Mais l'ennemi est sur ses talons. Bob prend son souffle, se concentre, et dépense un point de Stress pour ignorer sa peur du vide. Il lance donc son dé de Physique, pour un résultat de 4, et son dé d'Athlétisme, pour un résultat de 5. Bob ne peut pas prendre le 5, il a ignoré sa peur du vide, mais pas sa blessure. Il doit donc prendre le résultat de 4.

S'il l'avait voulu, Bob aurait pu ignorer sa seconde faiblesse (la blessure) pour prendre le résultat de 5, au prix d'un autre point à dépenser (Panache ou Stress).
}

\begin{multicols}{2}
\section{Panache, Stress et Traits}

Le panache et le stress sont deux types de points utilisables par les joueurs. Leurs conditions d'utilisations sont identiques, la seule différence entre ces deux types de points sont la façon dont ils se rechargent.

Il est conseillé d'utiliser des marqueurs de deux types (formes ou couleurs différentes) afin de symboliser les deux types de points.

\subsection{Panache}

Le Panache est distribué au joueur par le MJ suite à un bon roleplay, une bonne description, une bonne idée, ou tout autre élément favorisant une bonne partie. Une fois que le MJ a commencé à donner des points de Panache aux joueurs, ils peuvent les utiliser dès qu'ils le jugent nécessaire.

Toutefois, la réserve du MJ n'est pas illimitée. Il faut donc dépenser les points de Panache pour que le MJ puisse à nouveau les redistribuer. 

La réserve de points sur la table au début d'une partie dépend du niveau d'héroïsme. En effet, il faut placer sur la table, au début de la séance, un nombre de points de Panache égal au nombre de joueurs multiplié par le niveau d'héroïsme. Au début de chaque partie, chaque joueur commence avec 1 point de Panache.

\subsection{Stress}

Les points de Stress sont directement à disposition des joueurs. Tous les joueurs peuvent, quand ils le désirent, utiliser des points de Stress. Une fois les points de Stress utilisés, ils sont mis à disposition du MJ. Le MJ pourra alors les utiliser pour ses PNJ et les remettra ensuite dans la réserve générale.

La réserve de points sur la table au début d'une partie dépend du niveau de Tension. En effet, il faut placer sur la table, au début de la séance, un nombre de points de Stress égal au nombre de joueurs multiplié par le niveau de Tension.

\subsection{Utilisation}

Les points de Panache et de Stress ont différentes utilisations.

\begin{itemize}
\item Utiliser un trait (voir détail ci-dessous)
\item Ignorer une faiblesse
\item Utiliser un pouvoir occulte (Psy, Magie, ...)
\item Influencer la narration
\item Assister une action
\end{itemize}

\subsection{Utiliser un trait}

Un trait peut être utilisé lorsqu'il correspond à l'action entreprise. L'utilisation ou non d'un trait peut être discutée entre les joueurs et le MJ.

Les traits peuvent être ceux du personnage qui peuvent l'aider dans l'action qu'il tente, mais également des éléments Descriptifs de l'environnement (des débris pour se cacher, un tuyau par terre comme arme improvisée, un lustre pour échapper au bretteur ennemi, etc...) !

L'utilisation d'un trait nécessite la dépense d'un point de Stress ou de Panache.

Un trait utilisé permet de :
\begin{itemize}
\item Relancer tout ou partie d'un jet de dés
\item Faire exploser les dés : les dés indiquants leurs maximums sont relancés et cumulés.
\end{itemize}




\subsection{Influencer la narration}

Cette option doit être validée par le maître du jeu avant utilisation. Nous incitons toutefois les maîtres du jeu à l'autoriser car elle donne aux joueurs un peu plus de pouvoir et permettent parfois des rebondissements originaux. Bien sûr, à chaque utilisation, le MJ est libre de donner son véto.

Influencer la narration permet aux joueurs de dépenser des points de Panache ou de Stress pour compléter la description du MJ. Il ne s'agit pas de venir contredire ce que le MJ a déjà dit, mais bien de préciser des choses qu'il n'a pas encore dites. Selon les cas, le MJ peut demander au joueur de dépenser 1 (influence légère), 2 (influence moyenne) ou 3 (influence vraiment importante) points de Panache ou de Stress (ou des deux, par exemple 1 point de Panache puis 1 point de Stress pour compléter si le joueur n'a pas assez de points de Panache).

Un nouvel élément Descriptif de scène peut être créé suite à la description du joueur. Dans ce cas, la première utilisation que le joueur fera de ce nouveau Descriptif de scène est considérée comme gratuite.



\subsection{Ignorer une faiblesse}

Cette option permet au joueur de ne pas être pénalisé par une faiblesse ou un handicap. L'annulation ne vaut que pour l'action en cours : la faiblesse pénalisera à nouveau le joueur dès la prochaine action.

\subsection{Compétence complémentaire et Assistance}

Parfois, plusieurs personnages travaillent ensemble à la résolution d'un objectif commun. Dans ce cas, un des personnages va réellement exécuter l'action finale, on le nomme alors le "Meneur". Les autres personnages viennent alors juste apporter leur aide à ce meneur, avec leurs compétences propres.

Quand un tel cas de figure se présente, seul le meneur va effectuer le jet permettant de savoir si l'action est réussie ou au contraire échouée. L'aide des autres joueurs va permettre d'apporter des bonus augmentant les chances de succès. Ce bonus se présente sous la forme d'un (ou plusieur)s dé(s) supplémentaire(s) correspondant(s) au(x) dé(s) de compétence de la (ou des) personne(s) assistant le meneur. Pour bénéficier de cette aide, pour chaque personne assistant le meneur, 1 point de Panache ou de Stress doit être dépensé. Ces points peuvent être dépensés au choix par les assistants et/ou le meneur de l'action, de la manière qui les arrange.

Cette règle fonctionne également dans le cas de compétence complémentaire : c'est-à-dire quand une des compétences d'un personnage peut l'aider à réussir une action portant sur une autre compétence. Dans ce cas, c'est bien sur le joueur du personnage qui doit dépenser le point.

\end{multicols}

\exemple{Exemples d'utilisation de trait}{
\begin{itemize}
\item Le trait Forte-Tête peut être utilisé pour résister à une tentative de Persuasion. Par contre, il ne peut pas être utilisé pour gagner une course de vitesse (sauf si le joueur trouve vraiment une explication béton, dans ce cas, acceptez-la car il faut favoriser l'inventivité des joueurs).
\end{itemize}}

\option{L'utilisation d'un trait requiert de l'originalité !}{
Vous avez l'impression que vos joueurs se répètent et utilisent toujours le même trait, d'une manière simimaire ? Vous aimeriez plus de diversité, d'originalité ? Forcez vos joueurs à se renouveler !

La règle est simple : un trait donné ne peut pas être utilisé deux fois dans la même scène avec une justification identique. Si le joueur fait preuve d'originalité en donnant des justifications différentes, il peut bien sûr utiliser plusieurs fois le trait.
}


\exemple{Exemple d'influence sur la narration}{
\begin{itemize}
\item Le MJ décrit que le joueur, passant de ruelle en ruelle, tombe soudain sur un cul de sac. Le joueur peut alors dépenser 1 point pour indiquer qu'une échelle est posée le long du mur, et que, grâce à elle, il peut se sortir de cette impasse !
\\ % Sert à forcer un retour à la ligne
\item Coincé dans une cave, le joueur explique au MJ qu'il compte tenter d'ouvrir la porte avec une pièce de métal trouvée dans un recoin sombre de la pièce.
\end{itemize}
}

\exemple{Exemple : Assaut stratégique}{
    John est un "commandant" (d6) qui est "habitué à travailler avec de l'infanterie" (d10). Il mène un assaut face à un ennemi.

    Un vaisseau mené par Bob survole la zone et la scanne. Il redirige ses informations vers John. Bob utilise donc ses compétences en "senseurs" (d8) pour isoler les informations intéressantes et les envoyer à Bob.

    De son côté, John, fier de sa "grande expérience dans les tactiques d'infanterie" (d10), essaye de donner les ordres les plus pertinents.

    La compétence principale de cette action correspond à la compétence de commandement de John (d6). John décide de s'aider avec les données transmises par Bob. Il dépense un point de Panache. Il décide également de profiter de son expérience en infanterie et dépense un second point. Il va donc lancer sa caractéristique de Social (d4), sa compétence de Commandement (d6), sa compétence d'infanterie (d10) et la compétence de senseur de Bob (d8). Par contre, il aura déjà dépensé 2 points de Panache, ce qui représente au niveau d'héroïsme 2 le maximum dépensable sur une action.
}


\option{Inversez les rôles !}
{
    De temps à autre, une faiblesse semble pouvoir correspondre à une action et aider à sa réalisation. Des fois, nous sommes dans la situation inverse : un trait semble handicaper les joueurs dans la réalisation d'une action. Si tel est le cas, pourquoi ne pas permettre son utilisation !
	
	Lorsqu'un joueur veut utiliser une faiblesse comme avantage il faut :
	\begin{itemize}
		\item La faiblesse corresponde bien à l'action. Le MJ est seul maître de ce qu'il accepte ou refuse dans ce domaine.
		\item Que le joueur dépense 2 points de P/S : 1 pour l'inversion temporaire de rôle, 1 pour l'utilisation en tant que trait. Attention à la limite due au niveau d'héroïsme !
	\end{itemize}
	
	Lorsque le maitre du jeu (ou même le joueur !) veut utiliser un trait comme une faiblesse :
	\begin{itemize}
		\item La trait doit correspondre bien à l'action. Le MJ est seul maitre de ce qu'il accepte ou refuse dans ce domaine.
		\item Le joueur gagne 1 point de Panache pour cette inversion temporaire. La limite du nombre de points de Panache disponible n'est pas prise en compte pour cette attribution.
	\end{itemize}
}



