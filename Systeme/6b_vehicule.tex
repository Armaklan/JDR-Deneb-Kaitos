\chapter{Véhicules}

\begin{multicols}{2}

Les Véhicules vont être classés en deux catégories : 
\begin{itemize}
\item Les véhicules standards
\item Les véhicules personnalisés
\end{itemize}

Chacune des catégories réponds à des règles spécifiques.

\section{Véhicules Standards}

Ils n'ont pas d'identité propre et ne sont pas destinés à avoir un fort impact dans le jeu. Ce sont tous les moyens de transports qui servent pour une unique scène. Dans des univers non-technologique ils dépendent tous de cette catégorie.

Dans Fusina, ils sont gérés comme des équipements et apporte donc un dé supplémentaire à lancer.

\section{Véhicules personnalisés}

Ils sont unique et réèlloement lié au groupe ou à l'univers. Ils font des apparitions fréquentes dans les scénario du groupe. Ce peut-être par exemple le vaisseau spatial du groupe ou un mécha customisé par un personnage.

Ils sont représentés par trois caractéristiques : 
\begin{itemize}
\item Motorisation : utilisation du véhicule "à fond les manettes".
\item Systèmes : détection, communication, armements à distances, ...
\item Maniabilité : capacité de mouvement et de changement de direction.
\end{itemize}

Ils vont également être doté d'un certains nombres de traits et de faiblesse.

Lors d'une action engageant un véhicule, nous allons donc utiliser : Caractéristique et Compétence de l'utilisateur + Caractéristique du véhicule. En plus de ces propres traits, le joueur peut activer les traits de l'engin.

Un véhicule est également dotés de deux scores d'armures :
\begin{itemize}
\item Coque : protège les dispositfs dans le but d'éviter les dommages importants.
\item Protection : protège les passagers. Ce score est inutile quand ceux-ci sont entièrement couvert par le véhicule comme par exemple dans un vaisseau spatial.
\end{itemize}

Ces deux scores utilisent la règle de protection standard et représente le nombre de niveau de blessure que l'armure est capable d'encaisser chaque scène.

\subsection{Creer un véhicule}

Les véhicules possèdent un nombre de traits variable qui dépend de la puissance désiré :
\begin{itemize}
\item Véhicule faiblard : 2 traits de bases.
\item Véhicule standard : 4 traits de bases.
\item Véhicule puissant ou high-tech : 6 traits de bases.
\end{itemize}

Comme pour les personnages, il est possible de choisir une faiblesse pour prendre un trait complémentaire. 

Les traits sont ensuite reliés aux différentes caractéristiques ce qui permet de déterminer leur score.

Il n'y a pas de règle concernant l'attribution de la coque et de la protection. Ces deux scores doivent être attribuer en fonction de la taille et du type de véhicule.

\subsection{Améliorer le véhicule}

Il est possible de modifier l'engin pour le rendre plus performant, plus résistant, ... Pour le faire, les joueurs devront sacrifier des points d'expériences. Ils peuvent sacrifier uniquement les points de la phase 1. Les coûts en expériences sont les suivants :

\begin{itemize}
\item Nouveau trait : 5 points
\item Lié un trait à une caractéristique : 5 points
\item Coque ou protection supplémentaire : 3 points
\end{itemize}

\end{multicols}
