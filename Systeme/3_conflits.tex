\chapter{Les conflits}

\begin{multicols}{2}

\section{Types et déroulement des conflits}

Dans beaucoup de jeux de rôle, quand les Personnages Joueurs (PJ) rencontrent un Personnage Non Joueur (PNJ) agressif, généralement, cela se solde par un conflit. Mais ce conflit n'est pas forcément physique. En effet, tout dépend de ce que les joueurs souhaitent faire. S'ils interrogent le PNJ, cela pourra se régler en un conflit d'Intellect, pour représenter le duel de volonté entre l'interrogateur qui veut l'information, et l'interrogé qui ne veut la donner.

Qu'ils soient physiques, mentaux, sociaux, d'influence, nous gérons dans Fusina tous les conflits de la même manière, mais avec deux degrés de détails : les conflits courts et les conflits importants.

\subsection{Les conflits courts (contre les figurants)}

    Ces conflits sont les plus courants : il s'agit de toutes les situations où un personnage affronte (physiquement, socialement, mentalement, etc...) quelque chose (un ennemi, une tâche complexe,...). On ne se préoccupe que de l'action du personnage. En termes de jeu, le joueur indique au MJ l'action qu'il souhaite accomplir.

    En fonction de l'action souhaitée, le MJ détermine la difficulté, et fait faire un seul jet au joueur pour déterminer l'issue de la scène. En effet, les conflits courts sont là pour l'action, mais n'ont pas de réelle importance dans l'histoire, la scène est donc résolue en une fois.
    
\subsection{Les conflits importants}

    A l'inverse, le(s) grand(s) méchant(s) compte(nt) beaucoup dans l'histoire, ils ne peuvent pas apparaître uniquement 30 secondes ! Il y a également certaines scènes porteuses d'une forte intensité dramatique qui méritent d'être traiter plus longuement.

	Dans ce cas, les actions vont s'enchainer en faisant progresser les différents objectifs poursuivis par les intervenants. Un but est rarement atteint en une seule action. Il faut en général plusieurs réussite pour réaliser un objectif. 

	

\section{La résolution d'un conflit long}

\subsection{Définition des termes de bases}

\textbf{Conflit long} : Opposition entre plusieurs personnages ayant suffisamment d’importance dramatique pour être traité dans une scène complète.

\textbf{Objectif} : But poursuivi par un ou plusieurs protagonistes lors du conflit. L’objectif ne correspond pas à ce que le personnage veut faire sur une action donnée, mais à ce qu’il compte réellement accomplir dans le conflit. Exemple : Libérer la donzelle, capturer le grand méchant, détruire la pierre ancestral de Magamot, …

\textbf{Objectif opposé} : Deux objectifs sont dit opposé quand la résolution empêche la résolution de l’autre. Exemple : détruire la pierre ancestral de Magamot et la protéger, fuir quand le camp opposé essaye de nous capturer ou de nous tuer, …

\subsection{Déroulement}

Le conflit est découpé en round. La durée d’un round dépend du type de conflit :

\begin{itemize}
\item Un round de combat va durer une dizaine de seconde
\item Un round d’une bataille va durer plusieurs minutes
\item Un round social pourra durer de longue minute
\item Un round dans un combat d’influence va durer plusieurs heures, voir même plusieurs jours !
\end{itemize}

Lors d’un round, chaque protagoniste va avoir l’occasion d’effectuer une action. Les actions se résolvent selon la règle standard de Fusina. Quand une action va à l’encontre d’un personnage, celui-ci peut se défendre. L’action est alors résolue selon les règles d’opposition simple.


Chaque action effectuée par un protagoniste va être lié à un objectif. En général la détermination de l’objectif est faite directement par le MJ en fonction de la description du joueur. En cas de doute, le MJ peut biensur demandé son avis directement au joueur pour déterminer quel but il poursuivait.

\end{multicols}

\exemple{Les objectifs}{Matthias chercher à libéré la donzelle capturer par le grand méchant. Malheureusement de vils brigands se trouvent entre lui et la jeune femme. Il décide de foncer sur le brigand pour forcer le chemin. L’action de Matthias est une action offensive qui sera résolu en opposition avec les brigands, toutefois l’objectif de Matthias est bel et bien de libéré la donzelle, il fera donc progresser l’objectif.}

\begin{multicols}{2}

\subsection{Progression et difficulté des objectifs}

A chaque objectif du conflit le maitre du jeu va attribuer deux scores :
\begin{itemize}
\item Une difficulté de réussite : seuil à partir duquel l’objectif va être effectué.
\item Une difficulté d’échec : seuil à partir duquel l’objectif sera devenu irréalisable. En général on exprime ce chiffre en valeur négative.
\end{itemize}

Quand des objectifs sont opposés, ils sont évalués ensemble. La difficulté d’un des deux objectifs compte comme difficulté de réussite, l’autre comme difficulté d’échec.  Ces deux difficultés sont évaluées sur le même tableau que les difficultés standard.

\begin{itemize}
\item Action enfantine : 1
\item Action facile : 3
\item Action malaisée : 5
\item Action difficile : 8
\item Action héroïque : 11
\item Action quasi impossible : 13 ou plus.
\end{itemize}

Quand une action vient d’être résolue, on va mesurer l’impact sur l’objectif visé.
\begin{itemize}
\item Si l’action est une réussite,  celle-ci va faire progresser l’objectif jusqu’à, peut-être, le faire réussir. On ajoute alors (Marge de réussite / 2) au score de progression de l’objectif. Si le score de progression dépasse la difficulté de réussite, l’action est réussit.
\item Si l’action est un échec, l’objectif va reculer jusqu’à, peut être, échouer totalement. On retire alors (Marge de d’échec / 2) au score de progression de l’objectif. Si le score de progression descend en dessous de la difficulté d’échec, l’action est échouée.
\end{itemize}

\end{multicols}

\remarque{Sur la progression et son utilisation}{

Il est une chose qu'il faut absolument garder en mémoire lors de l'utilisation de ce système : les jauges de progressions ne sont qu'un outil destiné à aidé le maitre du jeu à évaluer où en sont les différents objectifs. Nous conseillons donc au maitre du jeu de :
\begin{itemize}
\item Ne pas annoncer les scores aux joueurs. Il vaut mieux leur décrire la situation globale du conflit et leur faire ressentir le résultat.
\item Ne pas utiliser de façon trop strict les jauges chiffrés : quand vous êtes capable de gérer le conflit sans vous encombrer de la valeur chiffré, faite le ! La jauge est un outil pour vous aider, il ne s'agit pas du cœur du système.
\end{itemize}

}

\begin{multicols}{2}

\subsection{Qualificatif de scène}

	L'environnement d'une scène donnée peut également offrir aux joueurs des éléments qu'ils pourront utiliser, ou au contraire subir. Ces éléments sont représentés par les "Qualificatif de scènes". C'est au maitre du jeu qu'incombe la lourde tâche d'informer les joueurs sur les possibilités de la scène. Bien sûr les joueurs sont encouragés à faire des propositions au mj en fonction de sa description ! Un qualificatif peut à le fois être utiliser comme faiblesse, et comme trait.

	Prenons quelques exemples :
	\begin{itemize}
		\item ÊTRE en apesanteur pour des personnages non habitués est un handicap. Ce handicap disparaitra bien sûr dès que le personnage y aura remédié d'une façon ou d'une autre (bottes magnétiques, scaphandre avec propulseur, ...).
		\item L'obscurité peut être à la fois un handicap (pour viser par exemple) et un trait (pour être discret).
		\item Un position dominante sur un champ de bataille est un trait en ce qui concerne la visée, ou la direction des troupes.
	\end{itemize}

\section{Blessures, handicap, et mort}

Dans les films (oui encore), il n'y a pas de jauge de santé au-dessus de la tête du personnage. Mais on peut voir ses blessures, on voit que chaque grosse blessure le handicape et on déduit donc s'il est encore en état ou pas d'agir, s'il est proche ou pas de la mort.

Selon les films, le héros peut subir moult choses avant de commencer à souffrir, dans d'autres il se retrouve dans une situation critique au premier assaut subi.

Nous avons donc souhaité reprendre ce principe dans Fusina, ce qui fait que chaque action pouvant causer une blessure (pas seulement les combats) donnera une faiblesse particulière que l'on notera dans l'encadré "Blessures" du personnage.

Elle est cependant semblable à une faiblesse classique à un détail près : la notion de gravité de la blessure, qui joue simplement sur la durée de récupération de la blessure.

Autre point important, c'est le joueur qui décidera de prendre une blessure. Ainsi chaque blessure aura un réel sens dramatique.

\subsection{Subir des blessures}

Un personne peut décider de subir une blessure pour augmenter ses chances de réussite. Dans ce cas, il bénéficie d'un bonus pour l'action en cours. Par contre, dès les actions suivantes il subira de plein fouet la blessure.

Le bonus dépend de la gravité de la blessure subit :
\begin{itemize}
\item Blessure Légère : +1
\item Blessure Importante : +2.
\item Blessure Mortelle : +4.
\item Mort : +6.
\end{itemize}

Durant une même scène, un personnage ne peut pas subir plus d'une blessure de chaque catégorie.

\subsection{Gravité des blessures}

La gravité indique quand la blessure est guérit. Un bon soin permettra de faire diminuer cette durée.

\begin{itemize}
\item Blessure Légère : Fin de la scène.
\item Blessure Importante : Fin du scénario.
\item Blessure Mortelle : Fin du scénario. La blessure laissera par contre une séquelle, une nouvelle faiblesse permanente qui devra être choisie en accord entre le maitre du jeu et le joueur.
\item Mort : D'une manière général, la mort est définit et amène le personnage à ne plus être jouable, que ce soit une mort physique ou social. 
\end{itemize}

Pour rappel, vous pouvez comme pour les faiblesses ignorer une blessure sur une action au prix modique d'un point de Panache ou de Stress.

Les blessures décrit ci-dessus ne s'appliquent pas uniquement au conflits physiques. Elles peuvent être utiliser dans tout type d'affrontement (Influence, Social, ...). Dans ce cas les blessures se traduisent par des Handicaps qui ont des effets et des durées similaires. Par exemple, un handicap d'Influence amènera le joueur à être coupé temporairement de ces contacts.

\end{multicols}

\exemple{Exemples de blessures}{

    Bob souhaite sauter par dessus une grille avec des barbelés pour entrer dans la banque. S'il ne fait pas une réussite totale, il se prendra des blessures en s'éraflant sur les barbelés.
    
    Une manière de procéder pour le MJ serait de dire qu'en cas de réussite partielle, il parvient à passer mais s'érafle légèrement, c'est donc une blessure légère. En cas d'échec partiel, soit il ne passe pas et se fait peu d'éraflures (blessure légère) soit il passe mais s'érafle profondément à plusieurs endroits aux pieds (blessure grave). En cas d'échec total il se blessera fortement et ne passera pas (blessure grave).

}
    
\exemple{Autre Exemples de blessures}{

    Bob doit affronter John, le vigile, une fois la barrière avec les barbelés, car il a crié en s'éraflant sur ces derniers. John sort sa matraque et demande à Bob de ne plus bouger, de se mettre à plat ventre, les mains sur la tête. Mais Bob décide de tenter de l’assommer. Bob décide de subir une blessure importante pour maximiser ses chances de résoudre l'affaire rapidement. Il effectue donc son jet et profite du bonus associé. Malheureusement pour Bob, John est bien meilleur combattant et arrive à se défendre, malgré le bonus. Au final, durant son assaut Bob se fait blesser gravement. John riposte, Bob blessé n'a aucune chance. Il s'effondre, assommé (car telle était l'intention de John).
}

\exemple{Un conflit dans son ensemble}{

Korgan le barbare, Matthias le chevalier, et Mareva la ranger sont à la recherche de Typhanie, une jeune paysanne disparu. Leur enquête les a menés dans le repaire de brigands. S’infiltrant ils ont découvert la salle où se trouve la jeune femme mais une grande majorité des brigands sont réunis. Le maitre du jeu estime que libérer la jeune femme est assez compliqué au milieu de tous ses brigands, il attribue une difficulté de 7. Paradoxalement les brigands sont chez eux, ils ont plusieurs échappatoire possible et ils y arriveront à -4. Typhanie décide de tenter sa chance et d’approcher discrètement, profitant de nombreux abris offert par la salle. Elle tente son jet et… zut ! Echec de justesse (1 points) ! L’objectif est donc plutôt mal parti (-1).

Le joueur décrit : J’approche, profitant des tables, tonneau, … Malheureusement, je suis tellement concentré sur les brigands debout que je ne vois même pas un brigand affalé par terre en train de cuver. Je me prends les pieds dedans et  du coup, fait un peu de bruit. J’arrive toutefois à me remettre à l’abri avant qu’ils ne regardent dans ma direction. Le MJ valide la description et rajoute : les brigands sont sur leurs gardes maintenant, quelqu’un se rapproche du lieu d’où vient le bruit. Korgan décide d’intervenir et se jette en avant en poussant un cri de guerre pour les impressionner. Ils espèrent leur faire peur et les désorganiser. Il fait son jette et réussit assez largement (de 4 !).

Le joueur décrit : Poussant mon cri de guerre je me dévoile ! Plusieurs brigands reculent impressionné, la plupart hésite, cherche de leurs yeux leur leader pour voir ça réaction. Je vois dans les yeux qu’ils connaissent ma réputation. L’objectif avance de 2 points et passe a +1. Le MJ valide la description et juge que la description de Korgan reflète assez l’avancement de la scène pour ne pas en rajouter. Matthias se lance à son tour en avant et charge les brigands, sont but n’est pas de tuer mais de crée un chemin pour permettre à ses compagnons de passer. Il fait son jet et réussit de 3 points, pas mal ! Il décrit : Profitant du doute que je lis dans leurs yeux, je charge ! Il n’était pas prêt pour ça et j’envoi valser les premier brigands que je rencontre. Désorganisé par la charge un chemin béant s’ouvre à mes compagnons.

Le MJ regarde l’avancé de l’objectif qui passe à 3/7, encore 4 point tout de même ! Il complète alors : en effet tu as commencé à faire une percée, malheureusement il y a encore des brigands sur le chemin ! Toutefois, ils sont là par pur hasard, vos actions ne leurs ont pas encore donné la chance d’établir une vrai ligne de défense. 


Un peu plus tard dans le conflit, Korgan repère son pire ennemi. Tant pis pour l’objectif, la vengeance d’abord ! L’ennemi n’est pas un guerrier, et le MJ juge que dans cette situation, les brigands ne chercheront pas particulièrement à le protéger. L’abattre est donc assez facile, il fixe une difficulté de 4 (nouvelle objectif) pour le mettre hors combat. Le joueur de Korgan estime qu’il ne veut pas que ça traine et joue le tout pour le tout ! Il annonce qu’il va prendre une blessure grave dans l’action pour avoir le précieux +4. Il fait son jet et… Marge de réussite 7 !! Soit un impact de 4 ! L’ennemi est mis hors combat du premier coup (objectif accompli).

Korgan décrit alors son action et sa charge, mais il décrit aussi comment, lors de celle-ci, la blessure grave lui est infligée. Il décide qu’un des brigands à profiter de son inattention pour lui porter un coup à la jambe, une vilaine blessure qui va le faire boiter. Pendant ce temps là, la réussite n’a pas été à la rencontre des compagnons de Korgan. Les brigands ont agit et personne n'a su les en empêcher ! Ils ont fait aussi aussi une réussit particulièrement forte et on fait reculer l'objectif de 4 points ! Ouch ! On en est à 0 ! Dernière le chevalier et la ranger on échoué leurs actions faisant descendre de 2 points chacun l'objectif. Aie !

L’objectif en est a -4, presque échoué ! Le MJ décide de refléter ça via la description suivante :
Un petit groupe de brigand est en train d’emmener la paysanne vers une petite sortie ! Vu l’organisation du repaire, si ils arrivent à l’emprunter ça sera perdu, ils arriveront à se sauver ! Là ils sont à 2 pas, il va falloir agir et vite !

}

\begin{multicols}{2}

\section{Et si les ennemis sont nombreux ?}

Dans la plupart des jeux de rôle, il y a deux types de règles de conflit, d'un côte les règles en conflit détaillé (quelques ennemis contre le personnage) où chaque ennemi agit, et les règles de conflit de masse. Généralement, les règles changent beaucoup et sont parfois complètement mal dosées, ce qui peut donner du 20 contre 2000, où, d'un jet, les 2000 meurent sous le joug menaçant des 20 pourtant moins bien entraînés et mal équipés.

En effet, souvent, on passe d'un combat détaillé avec plusieurs jets par round à un combat global résolu en un seul jet, et en général, les personnages, joueurs ou non, voient leur sort dépendre de ce seul jet de dés.

Dans Fusina, les petits groupes sont gérés par un dé bonus supplémentaire dépendant de l'avantage numérique du groupe sur le ou les personnage(s) et de la cohérence de ce groupe. Le dé se détermine en fonction de ces deux critères. Le dé de base est déterminé par l'avantage numérique, dé qui sera affecté par la cohésion de ce groupe, en bien ou en mal.

Voici le niveau du dé en fonction de l'avantage numérique du groupe sur le(s) personnage(s) :

\begin{itemize}
\item Léger avantage : d4.
\item Avantage moyen : d6.
\item Avantage important : d8.
\item Avantage écrasant : d10.
\end{itemize}

On prend en compte ensuite la cohésion du groupe :

\begin{itemize}
\item Groupe pas du tout coordonné, pas d'habitudes de fonctionnement en groupe : Le dé bonus baisse d'un échelon.
\item Groupe à cohésion classique, quelques belles ententes mais pas d'entraînement optimisant tout ça : On ne change pas le dé.
\item Groupe à forte cohésion, chaque membre du groupe à l'habitude de fonctionner dans le groupe : Le dé bonus augmente d'un échelon.
\end{itemize}

Cette gestion de groupe ne s'applique pas seulement à des conflits physiques mais à tous types de conflits, que cela puisse être un conflit social, mental ou d'influence...

\end{multicols}

\remarque{Et pour les groupes encore plus grands ?}
{
    Et bien, continuez à gérer cela comme des groupes de personnages !
    
    Après tout, l'échelle change, mais soit il s'agira d'un conflit "classique", soit l'un des camps n'a aucune chance et l'autre le surpasse complètement. Et à ce moment là, on peut gérer cela comme un conflit contre des figurants, mais à grande échelle. 
    
    Bref, il n'y a pas besoin d'une gestion différente des conflits !
}

\begin{multicols}{2}

\section{Conflits sociaux}

Imaginons que le joueur soit un politicien qui a eu plusieurs aperçus de l'existence d'une vie extraterrestre sur Terre. Il doit convaincre plus haut que lui de la menace, mais n'a que peu de preuves directes en dehors du fait d'avoir été témoin. Il parvient, en faisant jouer ses contacts, à obtenir une audience auprès du chef de l'état de son pays.

Son but est donc de convaincre le chef d'état d'agir et de prendre les devants sur cette menace. 

Dans les jeux de rôle "classiques", beaucoup de MJ vont jouer la scène au roleplay, et faire un petit jet à la fin de la scène pour voir si finalement le RP sert à quelque chose.

Et bien ici, non. Comme toute action importante dans Fusina, le résultat des dés (réussite ou échec) doit être décrit par le joueur. Ici, le MJ va indiquer si le joueur réussit ou échoue, partiellement ou complètement comme dans toute action normale.

Aux joueurs et au MJ d'arriver au consensus, mais voici des exemples de consensus possible pour cette interaction Policien-Président face à l'affirmation d'une menace extra-terrestre sans preuves directes :

\begin{itemize}
\item Échec total : Le chef d'état ne croira pas du tout le politicien, et va sûrement agir pour que ce politicien perde en crédibilité.
\item Échec partiel : Le chef d'état ne croira pas du tout le politicien, et va sûrement lui demander de revenir quand il aura des preuves.
\item Réussite partielle : Le chef d'état aura des doutes, et donner des ressources au politicien pendant une durée limitée, pour obtenir des preuves avant d'agir plus ouvertement.
\item Réussite totale : Le chef d'état croit complètement le politicien, et lancera toutes les procédures qu'il faut pour annihiler la menace.
\end{itemize}

Ensuite, le joueur explique l'attitude qu'aura le chef d'état au fur et à mesure du dialogue au MJ.

Enfin, les intervenants jouent la scène, en une seule fois, sans jets de dés au milieu, dont l'issue est celle décidée plus tôt avec le MJ.

\section{Créatures et Adversité}

Le rôle de maître du jeu est de décrire et de faire vivre l'univers, mais également les personnages qui le compose. Mais comment procéder quand le maître du jeu a besoin d'évaluer la compétence d'un personnage non joueur ? Comment procéder quand les joueurs doivent faire un jet en opposition ? Faut-il définir intégralement le personnage ? Pour la plupart des cas, la réponse est non ! 

Nous allons proposer ci-dessous plusieurs méthodes dépendant principalement du type de personnage.

\subsection{Les Figurants}

Les figurants sont des personnages que l'on croise dans la partie, mais qui n'ont aucun rôle réel à jouer. Les figurants ne représentent même pas un défi pour le personnage, ils sont juste là pour faire joli, être des pots de fleurs ! Par exemple, l'herboriste qui vend ses herbes et onguents sur le marché est un figurant !

Que se passe-il si un joueur rentre en opposition avec un figurant ? La réponse est simple, il n'y a pas d'opposition ! Le figurant est un élement du décor et doit être traité comme tel. Le joueur va donc réaliser son jet contre une difficulté fixée par le maître du jeu, exactement comme s'il tentait juste d'escalader un mur.

\subsection{Les sbires}

Les sbires sont un peu plus que des figurants. Ils n'ont guère d'importance dans le déroulement du scénario mais sont tout de même là pour représenter un défi à relever, mineur certes, mais un challenge tout de même. Les sbires ont en général besoin d'être en nombre pour représenter un danger.

Comment connaître leurs caractéristiques ou leurs compétences ? Pour un sbire, le maître du jeu va uniquement définir des caractéristiques, et une occupation. 

Les caractéristiques seront définies en répartissant 4 echelons dans les caractéristiques. Exceptionnellement le maître du jeu pourra glisser 1 échelon de dé d'une caractéristique vers une autre pour spécialiser davantage le sbire. 

L'occupation correspond à l'activité principale du personnage. Il s'agit en général de son métier. Lors d'une action, l'occupation permettra de déterminer le dé utilisé pour la compétence. 

\begin{itemize}
\item Si le jet correspond directement à l'occupation du sbire, alors la compétence vaut D8.
\item Si le jet est seulement en rapport avec son occupation, alors la compétence vaut D6.
\item Dans tous les autres cas, on considère que le sbire n'a pas de compétence associée.
\end{itemize}

L'équipement est pris en compte normalement. Toutefois, en général, tous les sbires d'un même groupe portent un équipement similaire (de même catégorie) et partagent donc les mêmes bonus. Les sbires sont fait pour être gérés en groupe à l'aide des règles précisées plus haut.

Concernant la dépense de Stress pour le maître du jeu, il est possible de dépenser uniquement 1 point pour un jet concernant l'occupation du sbire. Dans tous les autres cas, pas de stress pour les sbires ! 

Prenons un exemple : des brigands. Mes brigands vont avoir les valeurs suivantes : Physique : D8, Ame : D6, Social : D6, Intellect : D4, Influence : D4. Quand il s'agit de tendre une embuscade et de faire un peu de combat, mes brigands auront D8 en compétence. Pour fuir à travers la forêt, ils auraient un petit D6. Pour déjouer un piège mécanique, là, ils devront se contenter de leurs caractéristiques.

\subsection{Lieutenants}

Un lieutenant est le chef d'un groupe de sbires. C'est lui qui commande et organise sa petite troupe et la mène à l'action. Le lieutenant se génère et s'utilise de la même manière qu'un sbire à deux particularités près :

\begin{itemize}
\item Un lieutenant peut utiliser 2 points de Stress. Il peut les dépenser dans les jets directements liés à son occupation, mais aussi dans les jets en rapport.
\item Quand un groupe de sbires perd son lieutenant, il est à la déroute et travaille moins efficacement. Le dé de groupe perd donc un échelon.
\end{itemize}

\subsection{Les PNJ principaux}

Les PNJ principaux sont les personnages qui ont un vrai rôle à jouer dans l'aventure et qui sont l'égal, ou presque, des personnages. Ce sont les grands méchants, mais aussi les seconds rôles permettant de générer une ou plusieurs scènes intéressantes.

Pour les PNJ principaux il faut définir un niveau de tension. En général le niveau de tension vaut 2, voir 3 pour les PNJ vraiment dangereux ou importants. 

Pour la génération, le PNJ disposera donc de 4 + Tension échelons à répartir pour ses caractéristiques. On lui attribuera ensuite 2 occupations. L'occupation principale sera à d10, tandis que la deuxième sera à d8. Cela fonctionne ensuite comme pour les sbires et les lieutenants, les jets directement en rapport se font avec le dé de la compétence, les jets en rapport avec le dé de l'échelon en dessous, et on ne prend pas le dé de la compétence si l'action n'a rien à voir avec elle.

\subsection{Les Bêtes}

Par bêtes on entend toutes les créatures non-intelligentes et non-pensantes, que ce soit des animaux, des monstres, ou des créatures de légendes. Pourquoi faire une catégorie particulière pour les créatures ? Tout d'abord, les caractéristiques définies pour les personnages ne sont pas adéquates pour un animal. L'influence a-t-elle le moindre sens pour une bête ? De plus, les bêtes ne peuvent pas utiliser d'armes, il faut donc leur donner d'autres moyens de rester un défi.

Encore une fois nous parlons des bêtes censées présenter un réel défi : les autres bêtes seront gérées comme des figurants. 

Pour les bêtes nous allons donc définir trois caractéristiques :
\begin{itemize}
\item Physique : concerne toutes les actions nécessitant de la force, de l'endurance, de l'agilité, ou de la résistance.
\item Instinct : capacité à faire confiance à ses instincts, à ses perceptions.
\item Tenacité : capacité à lutter contre ses instincts, ses peurs, ses phobies. 
\end{itemize}

Leur génération est ensuite identique aux sbires, lieutenants, ou PNJs principaux.

\end{multicols}

\option{Je veux un PNJ plus puissant !}{
	Vous désirez créer un PNJ plus costaud ? 
	
	Nous vous proposons cette solution qui donnera à votre PNJ un plus large panel de possibilités, tout en le rendant plus intéressant. Vous pouvez ajouter à ce PNJ un champ de compétence supplémentaire, qui sera considéré comme une seconde occupation. En échange, vous devez lui ajouter une faiblesse qui pourra être exploitée par les joueurs. La valeur de la nouvelle occupation sera égale à la valeur des occupations principales, diminuée d'un échelon.
	
	Vous pouvez également augmenter d'un échelon une occupation existante. Le coût est identique, vous devrez rajouter une nouvelle faiblesse à votre création.
	
	Par exemple, je décide de faire un loup boosté aux hormones. De base le loup à un champ de compétence "Chasse" à d8. Je décide de lui rajouter "Affoler les animaux tels que les chevaux" en champ de compétence, prévoyant de faire chuter les cavaliers. Je peux donc la mettre à d6 (l'échelon en dessous de d8), en échange j'impose à mon loup une faiblesse "Phobie du feu".
}

\remarque{Génération partielle}{

Les règles ci-dessus permettent de générer des personnages non-joueurs en quelques minutes (sauf PNJ principaux). Mais avez-vous vraiment besoin d'une définition aussi précise ? Souvent une seule caractéristique vous sera utile pour les sbires !

Au final, nous vous conseillons de générer vos sbires et lieutenants à la volée ! Décidez des valeurs de chaque caractéristique ou équipement au moment où vous en avez besoin, pas avant. Dans mon exemple de brigand, au début j'aurai défini uniquement le Physique. Quand un joueur tente d'intimider, là je me pose la question du score en Âme (d6 ou d4 ?). Avant cela, pourquoi s'embêter ?

}



