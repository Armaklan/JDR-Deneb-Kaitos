\chapter{La Progression}

\begin{multicols}{2}

\section{Attribuer les points d'évolution}

Le système de progression de Fusina a pour objectif de refléter réellement ce que fait le personnage, comment il est perçu par le joueur, mais aussi par le maitre du jeu et les autres joueurs. Ce système permet également de faire un débrief sur la partie.

Les points d'évolution se répartissent donc en plusieurs phases.
\subsection{Phase 1 - Le choix du joueur}

Le joueur choisit ce qu'il désire faire évoluer sur son personnage. Il attribue un seul et unique point d'évolution. Ce point peut concerner un trait, des compétences, une faiblesse, ou un équipement.

\subsection{Phase 2 - Le choix du groupe}

Le groupe de joueur choisit ensuite un endroit où attribuer un point pour le personnage. Pour cette répartition on se base sur les actions du personnages durant la partie : Qu'est-ce qui a été important chez le personnage durant cette aventure ? Qu'est-ce que le groupe a retenu ?

\subsection{Phase 3 - Le choix du maître}

Le maître du jeu attribue à son tour un seul et unique point au personnage. Il peut se baser sur le vécu du personnage durant la partie, mais aussi sur sa propre volonté et sur l'intrigue globale de sa campagne.

\subsection{Phase 4 - L'oubli}

Avec accord du maître du jeu, un joueur peut décider que son personnage "oublie" un élement qui le définit (trait, compétence, équipement, ...). En échange le maître du jeu lui redonnera un élement de même valeur. Parfois il arrive que c'est le scénario qui impose une perte (comme un équipement volé ou détruit). La conséquence est la même, le maître du jeu redonnera au personnage quelque chose de valeur identique.

\subsection{Phase 5 - L'accomplissement}

Cette phase n'est pas systématique. Elle intervient uniquement quand un personnage a accompli quelque chose qui lui était propre (abattre un de ses ennemis, atteindre un de ses objectifs). Dans ce cas le maitre du jeu attribue au joueur 1 à 5 points d'évolution. Le nombre de points dépend de l'importance de l'accomplissement.

\section{Faire évoluer le personnage}

\subsection{Compétence}

\begin{itemize}
\item Acheter un d4 : 1 point
\item Passer de d4 à d6 : 2 points
\item Passer de d6 à d8 : 3 points
\item Passer de d8 à d10 : 4 points
\item Passer de d10 à d12 : 5 points
\end{itemize}

\subsection{Trait}

Acheter un nouveau trait coûte 5 points.

\subsection{Lier un trait}

Lier un trait à une caractéristique standard ou occulte coûte 5 points.

\subsection{Faiblesse}

Supprimer une faiblesse du personnage coûte 10 points.

\subsection{Equipement}

Il est possible d'acquérir des équipements particuliers avec des points d'évolution. Les coûts donnés ci-dessous sont indicatifs, le MJ est seul juge du coût qu'il estime juste pour un équipement particulier.

\begin{itemize}
\item Equipement à d6 : 1 point
\item Equipement à d8 : 3 points
\item Equipement à d10 : 5 points
\item Equipement à d12 : 10 points
\end{itemize}

Ces équipements doivent avoir une identité propre qui les rend uniques.

\end{multicols}
