\chapter*{Introduction}

\begin{multicols*}{2}

\section*{La crise Fédéral}

De nouveau, la guerre a éclaté... N'apprendrons-nous jamais de nos erreurs ? La longue période d'anarchie qui a suivi les départs des Ergios ne nous as visiblement pas suffit. L'alliance Kaitienne, dans sa soif de pouvoir, n'a pas tenu compte de l'avis de la Fédération. Celle-ci à exploser en morceau. Les anciens conflits rejaillissent comme si la paix n'avait jamais existé.

L'alliance Kaitienne a ouvert le feu mais tous guettaient la reprise des hostilités ! Il n'a pas fallu longtemps pour que les grands vaisseaux de métal de l'alliance Terrienne affrontent les nombreux Teldrims et leurs troupes galvaniser par la prêtrise du Protectorat.

Les Snagirs, jusqu'alors pacifique et diplomate ont soudain changé de méthode ! Le Consortium, la plus grande corporation Snagir vient de faire un coup d'état et de prendre les rennes de leurs gouvernements, forçant la lourde mécanique administrative à réagir et à passer à l'offensive.

Les Corporations, devenus multi-stellaire durant la période de paix, se sont lassées de la politique des gouvernements actuels. À la stupeur de tous, les principales corporations ont fondées le Bureau Corporatiste, un gouvernement sous leurs directions directes. Le Bureau Corporatiste s'est alors principalement opposé à l'Alliance Kaitienne. Ils ont déjà prit à l'alliance plusieurs systèmes, entrainant le déclin d'une puissance jusqu'alors dominante.

Et comme si tous cela ne suffisait pas, une nouvelle race est apparus, les "Eskador". Ces créatures sortis de nulle part ont ravagé une partie de Kaitos, l'ancienne capitale. Ils ont alors été 

\section*{La Marque des Ergios}

Au milieu de cette guerre, un évènement étrange perturbe les puissances : certaines personnes ont développées des pouvoirs hors-du communs. Les mondes connus avaient déjà entendu parler de pouvoirs psy mais ils n'ont rien de comparable. Les marqués, c'est ainsi qu'on les nomme car ils portent sur eux un étrange tatouage, un ornement qui apparaît quand ils font usage de leurs puissances. La Marque des Ergios... 

Ces porteurs sont des êtres hors du commun qui pourrait bien décider du sort des planètes fédérales. 

Le Bureau Corporatiste l'a bien comprit, l'amiral Vondrehen, chef de leurs armées, la porte. Il a formé l'unité Phénix, une petite escouade de force spéciale d'élite doté de ce pouvoir. Elle est très vite devenus une force que les ennemis des corporations ont apprit à craindre. L'Alliance Kaitienne a assimilé cette leçon dans la douleur.


\section*{Que joue t'on ?}

Les joueurs interprètes tous des porteurs de la Marque des Ergïos. Ce sont des personnages exceptionnels que ce soit grâce à leurs pouvoirs, ou par leurs compétences. Un pilote sera l'un des plus doué des mondes connues, un combattant fera partie de l'élite dont tous le monde à entendu parler, …

Dans Deneb Kaitos, les héros auront une réelle influence dans la tournure que prendra les évènements. Ils seront aux centres des intrigues, impliqués dans tous les évènements important. Selon leurs actes, les mondes connus pourront se réunir dans une démocratie éclairée, sombrer dans un totalitarisme militaire, ou au contraire disparaître et mourir devant les hordes d'Eskador qui déferleront sur la populace sans défense.

Seront-ils prêt à risquer leurs vies pour sauver cet univers ?

Pour faciliter l'approche du jeu, nous conseillons de commencer en tant que membres de l’unité Phénix, forces spéciales du Bureau Corporatiste. Cette approche permet de regrouper des personnages d'origines et de peuples différentes et offrent de nombreuses perspectives. Bien sûr, le meneur de jeu est encouragé à choisir une approche adapté à son groupe et à sa campagne. Jouer un petit groupe de Terrien luttant contre l'avancé de l'armée Teldrim, ou jouer un membre du puissant Consortium Snagir permet d'aborder l'univers de façon différente et tout autant intéressant.

\end{multicols*}
