\part{L'Adversité}

\chapter{Les Eskadors}

\begin{multicols}{2}

Les Eskadors sont une force à propos de laquelle on ne sait que très peu de choses. À vrai dire, une seule chose est sûre : ce sont nos ennemis. Sans crier gare, sans préavis, les Eskadors sont venus sur Kaitos et ont exterminé une bonne partie de la population, visant militaires comme civils sans aucune distinction. Ce sont des ennemis terrifiants dotés d'une technologie supérieure à la nôtre, des ennemis qu'il ne faut pas prendre à la légère.

\section{Apparence}

Les Eskadors sont des êtres humanoïdes dotés d'une double paire d'ailes semblables à des ailes de libellule. Leur peau est noire et possède de nombreux reflets verdâtres. Ils sont dotés d'une carapace semblable à de la chitine. Les Eskadors sont de taille variable : la plupart d'entre eux font entre 1m40 et 1m60. Certains spécimens très rares peuvent aller jusqu'à 2m. 

Les Eskadors utilisent une technologie biomécanique : les éléments de leur technologie sont intégrés à leurs corps. Il ne s'agit pas de simples greffes, la technologie est réellement assimilée par l'organisme et devient une partie du corps. Toutefois dans les mondes connus, leur technologie est réellement peu comprise. Sur certains aspects ils semblent en avance sur nous (voyage dans l'espace sans portail, communication sur de longues distances, ...) mais, toutefois, ils semblent faire un usage très restreint de la technologie.

\section{Organisation}

Il existe en réalité trois type d'Eskadors. Ensemble ils forment une ruche soumise à l'autorité d'un seul être.
\begin{itemize}

\item La Reine : la reine est un individu unique chez les Eskadors. Il n'existe qu'une seule et unique reine. Une nouvelle Reine naît juste avant la mort de l'ancienne, quelque soit la cause de son décès. La Reine constitue le cerveau de la ruche, son âme. Elle commande à l'intégralité des individus de l'espèce. La Reine peut mettre au monde une nouvelle reine, en donnant sa vie, ou celle d'une des dominantes. La Reine est capable de lire dans les pensées et de commander à toutes les dominantes, quelque soit la distance qui les sépare. Nul Eskador à part certaines des dominantes n'a jamais vu la Reine.

\item Les Dominantes : les dominantes sont des êtres féminins. Dotées d'une esprit indépendant capable de réflexion, elles sont toutefois soumises à la reine et ne peuvent aller contre sa volonté. Les Dominantes sont un relais au pouvoir de la Reine : la Reine commande aux dominantes, et les dominantes commandent aux ouvriers. À l'instar de la reine elles ont un réel pouvoir pour imposer leur volonté à leurs sujets. Toutefois, les dominantes ne peuvent se lier qu'à leurs enfants. Elles sont incapables de commander aux enfants d'une autre dominante. Les dominantes ne peuvent mettre au monde que des ouvriers. Les dominantes sont les individus les plus grands de ce peuple.

\item Les Ouvriers : les ouvriers sont les individus masculins de ce peuple. Ils n'ont pas d'esprit propre et sont incapables de faire preuve d'indépendance. Les ouvriers sont entièrement soumis à leur dominante et peuvent se sacrifier sans même hésiter sur son ordre. Si une dominante meurt, ses enfants sont réduits à l'état de légumes et meurent petit à petit, se laissant dépérir. Chaque dominante peut commander un bon millier d'ouvriers.

\end{itemize} 

\section{Et si la reine mourrait ?}

La Reine commande aux dominantes, et par l'intermédiaire de celles-ci, à tout son peuple. Que se passerait-il si la Reine mourrait ? Le peuple Eskador disparaîtrait-il ? 

L'évolution naturelle conduit à bien des choses, mais pas à mettre en danger un espèce entière. Au contraire l'évolution privilégie toujours la survie. La disparition de la reine porterait un coup sérieux aux forces Eskadors mais ne les mèneraient pas à l'extinction. Libérées de l'influence de la Reine, les dominantes seraient alors totalement indépendantes, le peuple Eskador serait fragmenté en de multitude de petites colonies, de petites ruches. 

Chaque ruche commencerait sûrement à s'attaquer à ses consœurs, chaque dominante s'opposeraient aux autres pour devenir la plus puissante. Durant ces conflits, les dominantes feront de nombreux enfants et gagneront en force. Elles se battront et de nombreuses dominantes seront tuées. Le combat continuera jusqu'au jour où une dominante sera devenue tellement puissante qu'elle saura alors imposer sa volonté à une autre dominante : une nouvelle reine est alors née.

\section{Les Eskadors, leurs mondes, et les mondes connus}

Que font les Eskadors dans les mondes connus ? Il y a des millénaires, les Eskadors ont été sauvés par les Ergios, ils étaient alors proches de l'extinction. En les amenant sur une nouvelle planète, les Ergios ont décidé de modifier le cours de l'évolution des Eskadors : ils ont alors créé le lien télépathique qui les lie tous. Une fois les gènes Eskadors modifiés, les Ergios ont laissé l'évolution finir le travail jusqu'à obtenir l'esprit ruche si propre à cette espèce. 

Durant la première grande guerre, les Ergios ont donné aux Eskadors de nouvelles technologies telles que les voyages en second espace. Puis, une fois la technologie intégrée, ils ont indiqué aux Eskador où se trouvait leur ancien ennemi : le peuple Vélïos. Attaqué par un ennemi issu de leurs légendes, les Vélïos se sont repliés sur eux-même quittant le combat contre les Ergios.

Une fois que les Ergios ont décidé de quitter définitivement les mondes connus, les Vélïos ont réussi à repousser l'envahisseur. Les Eskadors se sont alors repliés, pansant leur plaies sur leur monde natal. Puis, décidant qu'il fallait multiplier leurs mondes, ils ont commencé à se reproduire sur de multiples planètes. Ce sont les Vélïos qui ont mis un frein à leur expansion en reprenant le conflit. Toutefois, les Eskadors ont pu coloniser une dizaine de systèmes solaires.

Les Eskadors vénèrent toujours les Ergios. Ils considèrent que les peuples des mondes connus sont non seulement des alliés Vélïos, mais également les ennemis de leurs dieux. Pour cette raison ils font la guerre aux peuples des mondes connus et refusent toute négociation. Les Eskadors préparent donc de nombreuses attaques visant à conquérir les mondes connus et à exterminer leurs populations. Heureusement pour nous, les Eskadors frappent à l'aveuglette : ils ne savent pas quels systèmes solaires les peuples de la Fédération ont conquis. Ils ne nous comprennent pas non plus assez pour tirer partie de nos désaccords et de nos conflits.

\end{multicols}

\note{Un ennemi pour les unir tous !}{
    Les Eskadors sont un ennemi dangereux, capable de détruire définitivement les mondes connus. Cela fait plusieurs siècles qu'ils se préparent à la guerre, des siècles qu'ils essaiment et se reproduisent. Les mondes connus devront faire un choix : s'unir ou mourir. Mais les gouvernements de la Fédération en seront-ils capables ? Sauront-ils s'allier avant qu'il ne soit trop tard ?

    Dans ce conflit, les principaux atouts des Eskadors sont leur nombre et leurs forces. Pour l'instant ils ont quelques avantages technologiques, mais les experts des mondes connus travaillent déjà à les gommer. Mais les Eskadors attaquent en grand nombre et obéissent à tous les ordres de leur stratège à la perfection et sans hésitation. Pour gagner il faut que des milliers d'Eskador se sacrifient ? Ils le feront. C'est cet aspect inhumain qui rend les Eskadors si dangereux.
}

\chapter{Les Ergios}

\begin{multicols}{2}

Les Ergios ne sont pas de notre monde, ce sont des créatures du second espace, des êtres qui ont évolué pour devenir intelligents, et qui ont ensuite appris à rejoint notre dimension. Ils apparaissent comme des êtres humanoïdes fait de magie. En réalité, il s'agit uniquement d'une apparence qu'ils avaient prit l'habitude de prendre : les Ergios sont capables de modeler leurs corps.

Aujourd’hui les Ergios ont quittés les mondes connus, ils sont partis alors qu'ils remportaient la guerre. Pourquoi ? Même ces êtres très évolués peuvent connaître des troubles internes. Au moment où la guerre tournait à leur faveur dans la Fédération, une guerre civile a éclaté chez eux. Les Ergios ont décidé de disparaître pour régler leurs problèmes. Mais que se passera-t-il quand ils auront définitivement réglé leurs soucis ? Vont-ils rester dans leurs mondes ? Ou vont-isl revenir et finir la guerre qu'ils avaient commencé ?

\section{Les Ergios et la technologie}

La technologie Ergios est extrêmement avancée. C'est à eux qu'on doit le voyage galactique, c'est à eux que l'on doit la création de la Fédération. Les peuples des mondes connus ont côtoyé pendant des années les Ergios, mais au final leurs technologies sont toujours très peu comprises. Pourquoi ? La raison est simple : la technologie Ergios est extrêmement liée au Psy. Sans psy, impossible d'interagir avec elle. Actuellement, dans les mondes connus, seuls les psys portant la marque des Ergios sont capables d'interagir avec leurs systèmes.

\end{multicols}

\note{Les Ergios dans les mondes connus}{

Les Ergios ont disparu, emmenant avec eux leur savoir, leurs vaisseaux, et les bâtiments qui étaient vraiment les leurs. Toutefois, tous les Ergios ne sont pas partis, il en reste une poignée qui rôde dans les mondes connus en toute discrétion, infiltrés, ce sont en général des renégats qui n'ont pas voulu abandonner les peuples fédéraux. 

En dehors de cette poignée d'individus, il reste d'anciennes structures Ergios qui attendent d'être découvertes et explorées, des structures que les Ergios n'ont pu faire disparaître : au mieux ils ont décidé de les camoufler.

}

