\part{Prétire}

\chapter*{Sssigar}

\section*{Description}

Depuis sa plus tendre enfance, la communauté Snagir où il évoluait attendait beaucoup de lui. Il montrait une assurance et une intelligence peu commune. Mais tout ne se passe jamais comme prévu, ou comme on l'espère. Chacun ses petites faiblesses, ses petits défauts. Le gros défaut de Sssigar a été de s'intéresser à un domaine tabou : l'intelligence artificielle.

Sssigar a été comme obsédé par ce domaine de recherche. Les anciens ont essayé maintes fois de le convaincre d'arrêter, sans grand succès. Sssigar a continué ses recherches, voulant passer son épreuve d'âge adulte via son étude des intelligences artificielles. Durant ses recherches, Sssigar a découvert que des générations de Snagirs s'étaient concentrées sur ce sujet. Il a découvert qu'un scientifique du nom de Assganar avait réussit à créer une intelligence totalement artificielle, construisant une cité pour finaliser son projet.

Sssigar localisa la cité et s'y rendit. Ce fut le début de la catastrophe. Les IA créées par Assganar étaient toujours actives, sommeillant en l'attente de leur réveil. À l'arrivée de Sssigar, les IA se remirent en action. Elles comprirent vite que leur temps était venu. De multiples cités sous-marines se réactivèrent simultanément. Des chantiers se mirent en action pour construire des vaisseaux dédiés aux IA. Les intelligences essayèrent d'envahir la planète mère Snagir. 

Elles auraient bien pu réussir... Mais la guerre venant d'éclater dans les mondes connus, le Consortium était prêt à l'action. Devant le danger, le Consortium prit le pouvoir et intervint militairement. Dans le feu de l'action, personne ne put contester le coup d'état. Une fois les IA reprimées, il était trop tard, le Consortium était devenu le maître absolu du gouvernement Snagir. Les IA furent vaincues, mais plusieurs vaisseaux réussirent malgré tout à s'échapper. 

Devant la catastrophe engendrée par son acte, Sssigar quitta la planète Snagir, se forçant à l'exil. Il se sent non seulement responsable du retour des IA, mais aussi de la perte de liberté des Snagirs. Suite à la catastrophe, les autorités Snagirs le recherchent activement. S'il rentre un jour à la maison, nul doute qu'un procès l'attend. 

Et pourtant, depuis quelques mois, il semblerait que le Consortium souhaite le retrouver pour d'autres raisons. Un projet scientifique qu'ils aimeraient lui confier... Mais Sssigar se méfie du Consortium comme de la peste, il fera tout son possible pour le renverser. Autant dire que Sssigar se refuse à travailler pour eux...

Il y a peu, le Bureau Corporatiste a engagé Sssigar pour étudier un étrange implant Ergios. Le problème, c'est que l'implant fait maintenant partie de Melissa, un jeune humaine. Sssigar est chargé de récolter les résultats de diverses analyses, mais également de suivre Mélissa pour intervenir en cas de problème.

\clearpage

\begin{multicols}{2}

\section*{Caractéristiques}

\begin{itemize}
\item Physique : D6
\item Intellect : D10
\item Social : D6
\item Ame : D6
\item Influence : D6
\end{itemize}

\section*{Psy}

\begin{itemize}
\item Esprit de la machine : D4
\end{itemize}

\section*{Traits}

\begin{itemize}
\item Esprit de la machine (Intellect)
\item Puits de science (Intellect)
\item Connait les IA mieux que personne (Intellect)
\item Chercheur de renom (Influence)
\item A toujours le dernier mot (Social)
\item Prêt à tout pour ses recherches (Ame)
\item Amphibie (Physique)
\end{itemize}

\section*{Faiblesses}

\begin{itemize}
\item En exil pour avoir touché au sujet tabou
\item Totalement mégalomaniaque
\end{itemize}

\section*{Compétences}

\begin{itemize}
\item Électronique et informatique : D10
\item Bidouillages et réparations improvisés : D8
\item Débats enflammés : D6
\item Passer inaperçu dans la foule : D6
\item Pilotage : D4
\item Pistolet étourdissant : D4
\item Technologie Ergios : D4
\end{itemize}

\section*{Equipement}

\end{multicols}

\clearpage

\chapter*{Meisanne}

Meisanne est une Teldrim issue de la race des esclaves. D'emblée, la nature l'a doté d'un don qui aurait du lui être fatal : Meisanne possède des pouvoirs psy. Normalement, chez les Teldrims, les pouvoirs psy sont reservés à la prêtrise. Les esclaves n'ayant plus de dieu, ils n'ont plus de prêtres. Les esclaves ayant des pouvoirs psy sont donc massacrés dès que les pouvoirs se révèlent. À l'époque de la Fédération, des pressions avaient été faites sur le gouvernement Teldrim pour qu'il arrête ce type de massacre, mais en vain. Plusieurs millénaires ont fortement intégré cet acte dans la culture Teldrim.

Toutefois, heureusement pour Meisanne, elle a eu la chance de naître sur Alhena, territoire Teldrim où la loi est la moins implantée. De plus, ses pouvoirs se sont révélés uniquement sur la fin de son adolescence. Au moment où son peuple s'en est aperçu, elle fut capturée et battue. Atok qui était de passage en tant qu'escorte d'un prêtre Teldrim fut pris de rage devant ce spectacle. Il attaqua, fit un massacre, et quitta la place avec Meisanne.

Meisanne est très croyante. Pour elle, le dieu des Teldrims esclaves existe encore et ne tardera pas à revenir, et alors, ce sera la chute des prêtres. En attendant ce jour, elle a rejoint l'Antalgis et a déjà libéré plusieurs de ses frères. En moins de 3 ans elle s'est fait une petite réputation et a déjà du échapper à de multiples reprises à des fanatiques Teldrim.

Selon les rumeurs, le grand conseil Teldrim aurait décidé d'envoyer la Vakendar, un ordre d'assassins fanatiques, contre elle. Même sans ces assassins, Meisanne doit se méfier de tout Teldrim qui pourrait découvrir son secret et réagir selon ces preceptes religieux. 

Meisanne a également été contactée par un petit groupe, le Eshola. Ce groupe chercherait à recréer une prêtrise destinée aux esclave. Pour l'instant Meisanne n'a pas encore pris sa décision.

\clearpage

\begin{multicols}{2}

\section*{Caractéristiques}

\begin{itemize}
\item Physique : D8
\item Intellect : D4
\item Social : D6
\item Ame : D6
\item Influence : D4
\end{itemize}

\section*{Psy}

\begin{itemize}
\item Tellien : D8
\end{itemize}

\section*{Traits}

\begin{itemize}
\item Pas de félin (Physique)
\item Membre de l'antalgis (Influence)
\item La prêtrise, plus qu'un déguisement, une seconde peau (Social)
\item Le tellien est venu à moi comme quelque chose de naturel (Psy)
\item Disparaître est une nécessité, une question de survie (Physique)
\item Je ne craint plus la douleur, on m'en a tellement infligé (Ame)
\item Mon esprit est comme une lame acérée, le bois que je forge le démontre bien (Psy)
\end{itemize}

\section*{Faiblesses}

\begin{itemize}
\item Fait partie de la race des esclaves
\end{itemize}

\section*{Compétences}

\begin{itemize}
\item Acrobatie : D10
\item Comédie : D8
\item Discrétion : D8
\item Corps à corps : D4
\item Lance disque Teldrim : D4
\item Religion Teldrim : D4
\end{itemize}

\section*{Equipement}

\end{multicols}

\clearpage

\chapter*{Atok Iberak}

Les ancêtres d'Iberak étaient les meneurs du clan Iberak, un ordre de guerriers d'élite Vélïos plaçant au dessus de tout la force physique, l'honneur, et le courage. 

Le clan Iberak prône le combat en duel, un contre un. Combattre en supériorité numérique est considéré comme déshonorant. Combattre quelqu'un incapable de se défendre également. Fuir devant le combat tout autant. Par contre, accepter sa défaite face à un adversaire plus puissant est parfaitement honorable.

Au moment du repli des Vélïos sur eux-même, le clan Iberak était devenu extrêment puissant, presque aussi influent que l'impératrice. Les meneurs du clan se sont opposés au repli Vélïos. La majorité du clan est donc restée dans la fédération.

Aujourd'hui, le clan ne compte plus qu'une cinquantaine d'individus dans la fédération. Leurs membres sont encore très appréciés dans les forces armées, toutefois. Les guerriers Iberak font encore partie de l'élite des forces de combat rapproché. Toutefois, ils ne suivent que rarement les ordres, et leur refus de combattre autrement qu'en duel pose de nombreux problèmes aux stratèges.

Atok est encore un membre influent du clan, toutefois, contrairement à ses ancêtres, il n'en est plus le chef absolu. Les membres du clan vivent maintenant de façon indépendante. Chaque membre a toutefois passé un pacte d'assistance et de non-aggression.

La dernière fois que le clan s'est réuni intégralement, il avait pour but d'aider un clan nomade face aux assauts des troupes terriennes. Les Iberaks ont fortement contribué à la défaite des forces humaines. Depuis, une prime repose sur la tête de chaque membre du clan.

Depuis que l'empire Vélïos a recommencé à prendre contact avec la Fédération, des rumeurs sont arrivées aux oreilles d'Atok. Ces rumeurs parlent d'assassins Vélïos qui auraient été envoyés contre les Iberaks par l'impératrice, qui aurait encore peur de l'influence potentielle des Iberaks.

Il y a presque 3 ans, Atok a sauvé Meisanne des griffes de son peuple qui s'apprêtait à la massacrer. Il tua plusieurs Teldrims pour permettre à Meisanne de fuir. Autant dire que Haltane est maintenant une planète à éviter.

\clearpage

\begin{multicols}{2}

\section*{Caractéristiques}

\begin{itemize}
\item Physique : D12
\item Intellect : D4
\item Social : D4
\item Ame : D8
\item Influence : D6
\end{itemize}

\section*{Psy}

\begin{itemize}
\item Contrôle du corps : D4
\end{itemize}

\section*{Traits}

\begin{itemize}
\item Griffes dorsales (Physique)
\item Membre influent du clan Iberak (Influence)
\item L'honneur passe par dessus tout (Ame)
\item Formé avec les meilleurs guerriers (Physique)
\item Capable de récupérer même des pires blessures (Physique)
\item En moi coule un secret du clan : la Liqueur de la mort (Physique)
\item Rien ne m'arrêtera jamais sinon la mort (Ame)
\end{itemize}

\section*{Faiblesses}

\begin{itemize}
\item Le clan Iberak a autrefois menacé l'influence de l'Impératrice Eternelle. Elle nous en veut toujours.
\item Mon sens de l'honneur est complexe : il m'interdit notamment d'écraser mon adversaire sous le nombre.
\end{itemize}

\section*{Compétences}

\begin{itemize}
\item Corps à corps : D10
\item Athlétisme : D8
\item Intimidation : D6
\item Déplacement en apesanteur : D6
\item Acrobatie : D4
\item Faire valoir son honneur et son nom : D4
\item Commandement : D4
\end{itemize}

\section*{Equipement}

\end{multicols}

\clearpage

\chapter*{Mélissa Whistmer}

Mélissa était une nomade, mais la station nomade où elle vivait était trop petite pour son ambition. Elle a rejoint l'alliance Teriennne pour être formée comme espionne. Sa carrière allait bon train. Promis à une montée fulgurante dans la hiérarchie terrienne, elle enchaînait les missions jusqu'à la mission sur Kaitos. La mission de trop...

Pourtant, la mission paraissait simple. S'infiltrer dans une des cellules criminelles des CCG, enquêter sur un objet d'origine Ergios aux mains des criminels, le récupérer, et revenir. Ça aurait pu être simple. Mais il y a une chose que ses supérieurs n'avaient pas prévu : la technologie Ergios est imprévisible.

Un implant, un foutu d'implant ! L'engin Ergios était un implant. Son infiltration a trop bien marché. Elle aurait du se méfier. Mais avant même qu'elle n'ait eu le temps de réagir, l'implant était dans son corps. Autre chose que ces supérieurs avaient sous-estimé, la cellule criminelle était liée à un groupuscule vénérant les Ergios : l'Aube Pourpre.

Il lui restait trois choix :
\begin{itemize}
\item Rejoindre l'alliance terrienne et se faire disséquer par les scientifiques
\item Rester dans l'Aube Pourpre, et nul ne sait ce que cette secte de tarés allait attendre d'elle.
\item Fuir...
\end{itemize}

Mélissa s'est réfugié un temps chez les Nomades, profitant de ses anciennes relations. Mais la fuite est difficile. De multiples forces la poursuivent sans arrêt, cherchant à la récupérer, elle et son implant.

Il y a peu, sur la station spatiale Hellenus, Mélissa a failli se faire capturer. Elle ne doit son salut qu'au fait que la station a été prise en feu croisé entre la force désirant la capturer, et une force inconnue désirant capturer un certain Andreï. Les deux forces étaient visiblement antagonistes et ont ouvert les hostilités. Andreï et Mélissa ont réussi à fuir à bord d'un vaisseau marchand qui s'est avéré aux mains du Bureau Corporatistes. Le Bureau Corporatiste a fait une proposition à Mélissa : il s'occupe de brouiller ses traces, en échange Mélissa travaille pour eux et autorise quelques analyses non-intrusives de son implant. 

\clearpage

\begin{multicols}{2}

\section*{Caractéristiques}

\begin{itemize}
\item Physique : D6
\item Intellect : D8
\item Social : D8
\item Ame : D6
\item Influence : D6
\end{itemize}

\section*{Psy}

\begin{itemize}
\item Contrôle de la gravité : D4
\end{itemize}

\section*{Traits}

\begin{itemize}
\item Sens de la tromperie (Social)
\item Sens du bricolage (Intellect)
\item Porteuse d'un implant Ergios (Ame)
\item Femme aux milles identités (Social)
\item Possède un réseau de renseignement étendu (Influence)
\item Sa mémoire est presque sans faille (Intellect)
\item A suivi un entrainement intensif de l'armée (Physique)
\end{itemize}

\section*{Faiblesses}

\begin{itemize}
\item Paria de la civilisation
\item De nombreuses forces en veulent à mon implant
\end{itemize}

\section*{Compétences}

\begin{itemize}
\item Arme de tir : D8
\item Se fondre dans une identité : D8
\item Séduction : D8
\item Athlétisme : D6
\item Engin de communication et d'espionnage : D6
\item Montre moi comment faire, je reproduirai : D4
\end{itemize}

\section*{Equipement}

\end{multicols}

\clearpage
