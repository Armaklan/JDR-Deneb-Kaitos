\part{Scénario - I believe i can fly !}

Ce scénario est prévu pour une petite équipe travaillant pour le Bureau Corporatiste. Cela dit il n'est pas très difficile à adapter pour une autre force (sauf l'alliance Kaitéenne vu que l'action se déroule chez eux). La plupart des personnages pourront tirer leur épingle du jeu sans trop de difficulté.

Il s'agit d'un scénario assez équilibré présentant une phase d'enquête, et une phase d'action. L'équilibre peut être fait par le maître du jeu en fonction du groupe. Le scénario peut être joué par des personnages débutants moyennant une petite explication des principaux gouvernements des mondes connus. Il ne dépend et ne révèle aucun secret du jeu.

Le principal atout de ce scénario est le cadre de jeu qui permet d'offrir des scènes assez uniques en leur genre. Son principal défaut est de commencer par une phase d'enquête ! Je vous conseille de commencer par une scène d'action "hors-scénario", histoire de défouler vos joueurs et de les plonger dans l'ambiance.

\chapter{Synopsis}

\begin{multicols}{2}

Le Bureau Corporatiste a réussi à intercepter un message destiné au Consortium Technologique Snagir. Ce message émanait d'un scientifique humain nommé Girard Montigny, particulièrement connus pour ses travaux sur les générateurs d'énergie au Sagium. Selon le message, Girard aurait fait une découverte importante en étudiant une épave Ergios écrasée depuis des siècles sur Denos 6. L'épave n'avait jamais été découverte auparavant et présente en elle-même un trésor rare. Girard demande au Consortium d'envoyer quelqu'un pour négocier avec lui sa découverte, et pour ensuite l'extraire de la planète.

Le Bureau Corporatiste charge les joueurs d'aller négocier avec M. Montigny pour le convaincre de travailler pour le Bureau Corporatiste. En dernier recours, les personnages devront enlever Montigny et le ramener en secteur corporatiste. Denos se trouve en plein secteur de l'alliance Kaitéenne, il est difficile pour le bureau corporatiste d'y intervenir. Les joueurs seront donc livrés à eux-même. Pour se rendre sur place, les personnages partiront avec une croisière de luxe qui se dirige justement sur Denos 6.

Une mission assez standard et qui demandera aux joueurs un peu de doigté. Sauf que quelque soit leur plan, un imprévu va venir tout chambouler ! En plus d'agents du Consortium, deux autres forces sont sur place ! L'Aube Pourpre a envoyé une petite milice chargée de prendre possession du complexe et de capturer Montigny. Sur place, une armée de résistant s'apprête à attaquer également. Leurs objectifs ? Démarrer leur mouvement révolutionnaire dont le but est de chasser l'alliance Kaitéenne pour remettre sur le trône l'ancienne monarchie Dénosienne.

\end{multicols}

\chapter{Les lieux}

\begin{multicols}{2}

\section{Le système Denos}

Denos est un système solaire appartenant à l'alliance Kaitéenne. En lui-même il n'a que peu d'importance stratégique : pas de ressource rare, pas de position stratégique vis à vis des Tellurias, ... Toutefois Denos est l'un des systèmes restant aux mains de l'alliance Kaitéenne et suite à leurs dernières défaites, l'alliance fait attention à ne pas reculer davantage. 

Heureusement pour les personnages (ou pas), l'alliance est en train de regrouper ses forces sur Ash'Nemba afin de mener une offensive contre Kaitos. La présence militaire dans le système est donc relativement réduite au moment des faits.

Denos possède 7 planètes différentes dont deux sont considérées comme habitables : Denos 3 et Denos 4. Denos 3 est fortement peuplée. Du temps de la monarchie c'était une planète riche qui mettait en avant le luxe et la culture. Aujourd'hui la société de Denos 3 s'enfonce dans la déchéance... Denos 4 est assez peu peuplée, mais son sol est particulièrement fertile et constitue un terrain de choix pour l'agriculture. Denos 4 est donc une planète de fermiers.

À l'extérieur du système, bien au delà de Denos 7, une station spatiale s'occupe de récolter les minéraux présents dans un champ d'astéroïdes. Ce que l'alliance Kaitéenne n'a pas encore remarqué, c'est que les nomades avaient secrètement transformé leur station spatiale en chantier spatial couplé à une usine d'armement. Les nomades locaux soutiennent la rebellion en lui fournissant des armes. Ils prévoient également de l'appuyer par un soutien spatial dès que suffisament de vaisseaux seront prêts.

\section{Denos 6}

A première vue, Denos 6 est une planète sans grand intérêt. Elle ne possède aucune ressource rare, est toujours balayée par un vent fort qui empêche toute agriculture à l'air libre, et sa haute atmosphère est particulièrement dense, ce qui rend difficile détection et communication.

Toutefois l'ancienne monarchie Dénosienne, avant la conquête par l'alliance Kaitéenne, avait su exploiter la planète. Ils ont créé un centre de vacances de luxe offrant non seulement des paysages totalement naturels et d'une beauté à couper le souffle, mais aussi la possibilité de pratiquer un sport rare : le delta-plane. En effet le vent constant de la planète permet d'offrir des sensations uniques à ceux qui osent se lancer dans le vide avec leur planeur.

La planète est donc vide à part une dizaine de stations flottant dans les nuages de la planète. Deux de ces stations ont un rôle particulier. La station 1 placée au centre sert d'astroport accueillant les navettes remplies de vacanciers et envoyés par les vaisseaux de croisière qui se placent en orbite. Les voyageurs sont ensuite dispersés à travers les autres stations à l'aide de "planeurs de transport", de grands véhicules utilisant le vent pour se déplacer de station en station, et pouvant transporter une bonne centaine de passagers.

La station 5 n'accueille par contre pas de voyageurs. Cette station sert de tour de contrôle et de relais de communication entre les différentes stations. Elle contient également les troupes de sécurité du complexe (en effectif assez réduit...).

Les huit autres stations sont uniquement destinées aux vacanciers. Chaque station peut accueillir jusqu'à 500 personnes. Malgré la période de guerre, le trafic vers ce centre de vacances n'a que peu diminué. L'alliance Kaitéenne a décidé de ne pas gêner le trafic vers les stations de vacances quelque soit sa provenance.

\end{multicols}

\chapter{Personnages et Forces en présence}

\begin{multicols}{2}

\section{Saganosss, agent du Consortium}

Saganosss est un Snagir sans grand trait distinctif. De petite taille, pas particulièrement costaud, il se fait passer pour un mécanicien Snagir qui profite d'une croisière qu'il a mis des années à se payer. Saganosss est un snagir jovial et communicatif, comptant plus sur son éloquence que sur sa force ou son habileté.

Bien sûr, Saganosss s'attend à une mission de routine et n'est pas préparé pour les conflits armés. Toutefois, c'est un homme, euh, un lézard malin et astucieux ! Il essaiera d'agir dans la plus grande discrétion et la plus grande finesse. S'il peut se servir des personnages pour atteindre son but, il le fera.

\section{La princesse Illiana, Malak, et les Rebelles}

Illiana est l'héritière de la couronne de Denos. Ses parents ont été capturés par l'alliance Kaitéenne lors de la conquête de Denos, personne n'a entendu parler d'eux depuis. Illiana a échappé à la capture et a rejoint un groupe de résistants. Sûre d'elle, hautaine, elle n'a que très peu de connaissances en stratégie, mais ses hommes lui sont très fidèles et sont prêts à tout pour elle.

Illiana est protégée par Malak, un guerrier Vélïos du clan Iberak. Ce Vélïos protège Illiana depuis plusieurs année et s'est vraiment pris d'affection pour elle. Malak est un vétéran qui a vu de nombreuses batailles. Il essaye de tempérer un peu le caractère d'Illiana et de la conseiller concernant les tactiques à utiliser.

Les rebelles sont une cinquantaine et tous regroupés sur la même station. Ils sont plutôt bien armés grâce aux nomades et obeïssent à Illiana et à Malak. Les rebelles ne voulant pas faire de dommages colatéraux sur les civils, ils sont équipés d'armes ayant un mode paralysant. Les rebelles ont l'avantage de bien connaître le terrain et de pouvoir attirer la sympathie des civils originaires de Denos.

\section{Les espions de l'Aube Pourpre}

Les espions de l'Aube Poupre sont sur les lieux depuis plusieurs jours. Ils auront déjà la plupart des renseignements sur Montigny et sur ses aller-retours. Durant leurs recherches, il est possible que les personnages se rendent compte que d'autres ont cherché à avoir les renseignements avant eux. Il sera toutefois très difficile de débusquer ces espions.

\section{Les forces de l'Aube Pourpre}

Quand les personnages arrivent sur Denos 6, un vaisseau de l'Aube Pourpre pénètre dans le système. Ce vaisseau dispose des dernières technologies dans le domaine de la furtivité et va approcher en toute discrétion de la planète. Une fois en orbite ils vont contacter leurs espions pour en savoir plus sur la situation. Ils commenceront à intervenir a peine une heure avant le retour de Montigny. Le vaisseau contient 200 soldats qui vont débarquer dans les stations à l'aide de navettes. La vingtaine de soldats de l'alliance Kaitéenne ne pourra rien faire contre eux. 

L'Aube Pourpre va, en premier lieu, prendre le contrôle de la station 1 et de la station 5. Elle espère ainsi empêcher toute communication vers les autres planètes du système, et empêcher toute fuite. Une fois la station 5 sous contrôle, ils enverront une équipe prendre la station 7 pour attendre Montigny.

\section{Les gardes de l'Alliance Kaitéenne}

Au début du scénario, les gardes de l'Alliance Kaitéenne sont principalement concentrés sur la station 5. Tant que les personnages ne se font pas remarquer, ils ne devraient pas avoir de problèmes avec eux. Quand l'Aube Pourpre attaque, la garnison de la station 5 se fait balayer sans trop de problèmes. Il y a peu de chances que des renforts de l'Alliance Kaitéenne arrivent avant la fin du scénario.

\end{multicols}

\chapter{Idées de scènes}

\begin{multicols*}{2}

Les scènes qui sont présentées ici sont des idées, elles ne sont pas obligatoires ! À vous de voir quelles scènes vous voulez mettre en place en fonction de vos envies, de votre groupe, et de vos personnages. N'hésitez pas bien sûr à rajouter vos propres scènes, ce scénario prendra toute sa saveur si vous le personnalisez ! N'oubliez pas non plus de ne pas vous enfermer dans ce que vous avez prévu.

\section{La croisière}

Ah, la croisière de luxe ! Les personnages aimeront sûrement profiter des attractions proposées sur ce superbe yacht tous frais payés : boissons raffinées, salle de jeu, piscine, ... La belle vie ! Toutefois, il passera toujours dans la tête d'un de vos joueurs de mener son enquête sur les autres passagers ! Et il aurait raison ! Sur cette même croisière se trouve l'agent du Consortium Snagir : un petit Snagir qui est tout sauf impressionnant, très passe partout en fait... Quoiqu'il arrive, lui il mènera sa petite enquête ! À voir si vos joueurs seront suffisament malins pour le démasquer et pour passer inaperçus ;)

Cela dit, ne vous éternisez pas sur cette scène : trop de lenteurs au début risque de tuer votre partie.

\section{Trouver M. Montigny}

Là commence la véritable enquête ! Cette scène z deux interêts :
\begin{itemize}
\item Décrire les lieux : des stations prévues pour les touristes uniquement, les moyens de transport, ... Essayez de donner une identité propre à chaque station que les personnages visiteront ! Une station donne sur les montagnes et propose un décor "alpin". Une autre station donne sur l'océan et donnera une ambiance "plage et sable chaud" avec possibilité de descendre se baigner ! Une troisième station pourrait avoir le thème des fonds marins et être remplie d'aquariums avec les poissons locaux, ...
\item Mener l'enquête sur le docteur Montigny : les joueurs pourront apprendre que le docteur n'est pas dans le complexe. Il est équipé d'une navette d'expédition et va sur la surface de la planète. Il revient régulièrement dans la station numéro 7 et devrait revenir d'ici une journée maximum !
\item Prendre connaissance sur les forces en présence. En enquêtant sur le docteur Montigny les personnages pourraient bien comprendre que quelque chose se trame ! Dans le complexe rôdent quelques espions de l'Aube Pourpre. Sans agir, eux aussi enquêtent sur le docteur Montigny ! L'agent du Consortium est là aussi pour la même raison. Mais le plus louche, ce sont ces hommes qui font pénétrer d'étranges bagages (des armes), et qui semblent obeïr à cette jeune femme (la princesse Dénosienne).
\end{itemize}

\section{Une seule raison, l'invasion !}

Après un long moment d'enquête, les joueurs apprennent que Girard revient vers le complexe et va attérir dans la station 3. Bien sûr, ils ne sont pas sur la bonne station ! 

Soudain, c'est la coupure des communications. Un message informatif est diffusé dans la station indiquant que jusqu'à nouvel ordre, tout transport ou toute sortie en deltaplane est annulée. 

Quelques minutes plus tard, c'est le courant qui est coupé. Les systèmes de secours s'allument. Un message demande à chacun de rester là où il se trouve et de ne plus bouger. Évidemment, les gens commencent à se poser des questions. Une force armée apparaît et commence à regrouper tout le monde dans l'un des principaux Salons. Ils se présentent comme les représentants de l'armée Dénosienne. Ils ne donneront toutefois pas plus d'explications. Ils demandent aux voyageurs de rester calmes en leur promettant qu'il ne leur sera fait aucun mal.

En les questionnant les joueurs apprendront qu'ils ne sont pas responsables de la coupure des communications qui les intrigue eux aussi... 

Comment les joueurs vont-ils s'en tirer ? Ils peuvent fuir, ou au contraire essayer de négocier.

\section{Le delta-plane, rien de tel pour partir à l'abordage !}

Comment circuler entre les stations ? Deux possibilités ! Ils peuvent voler un transport : facile mais pas très discret. Ils peuvent aussi emprunter des deltaplanes et se lancer dans l'inconnu en direction de la station 3 ! Sauf que le voyage ne va pas être de tout repos. 

Les deltaplanes ont la forme de deltaplane standard, munis toutefois de plusieurs appareils électriques : un radar courte portée, une sorte de GPS indiquant la position du deltaplane vis à vis d'une station en particulier, un moteur anti-gravité de secours, mais aussi un générateur d'ultrasons pour faire fuir les prédateurs volants locaux. Et s'ils ne se sont pas renseignés à l'avance, ils n'ont aucun moyen de le deviner ! De quoi faire une petite bataille aérienne avec de grands prédateurs volants style ptérodactyles.

Une fois le problème des prédateurs volants réglé, il faut encore attérir dans la station. Bien sûr sur les plateformes d'atterrissage il y a quelques soldats prêts à en découdre. À qui sont ces soldats ? À l'aube pourpre ! Ceux-ci ont pris le contrôle de la plateforme et cherchent à capturer le docteur. Celui-ci ne pourra pas leur échapper bien longtemps.

Pour la petite histoire, tout le scénario est issue de cette scène dont j'avais eu l'idée.

\section{Extraction de Montigny}

Une fois sur place, la situation n'est guère glorieuse... La station est sous contrôle de l'Aube Pourpre et déborde de soldats prêts à ouvrir le feu. Le docteur est tombé entre leurs mains, l'Aube Pourpre se prépare à l'évacuer en direction de la station 5, qui est sous leur contrôle. Heureusement, les forces de l'Aube Pourpre affrontent celles des rebelles Dénosiens, ce qui crée une diversion plutôt bonne.

Comment les joueurs vont-ils mettre la main sur Montigny ? Comment vont-ils évacuer finalement ? Voilà encore un dur défi à relever ! 

\end{multicols*}
