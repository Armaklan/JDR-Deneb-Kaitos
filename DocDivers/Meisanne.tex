\documentclass{DenebClass}

\title{Deneb Kaitos}
\author{Lionel "Armaklan" ZUBER}
\begin{document}

\chapter*{Meisanne}

Meisanne est un Teldrim issu de la race des esclaves. D'emblée, la nature l'a doté d'un don qui aurait du lui être fatale : Meisanne possède des pouvoirs psy. Normalement, chez les Teldrims, les pouvoirs psy sont reservé à la prêtrise. Les esclaves n'ayant plus de dieu, ils n'ont plus de prêtres. Les esclaves ayant des pouvoirs psy sont donc massacrés dès que les pouvoirs se révèlent. A l'époque de la Fédération, des pressions avaient été faite sur le gouvernement Teldrim pour qu'il arrête se type de massacre, mais en vain. Plusieurs millénaire ont fortement intégré cet acte dans la culture Teldrim.

Toutefois, heureusement pour Meisanne, elle a eu la chance de naitre sur Alhena, territoire Teldrim où la loi est la moins implanté. De plus, ces pouvoirs se sont révélés uniquement sur la fin de son adolescence. Au moment où son peuple s'en est aperçu, il fut capturé et battu. Atok qui était de passage en tant qu'escorte d'un prêtre Teldrim fut pris de rage devant se spectacle. Il attaqua, fit un massacre, et quitta la place avec Meisanne.

Meisanne est très croyant. Pour lui, le dieu des Teldrims esclavent existent encore et ne tardera pas à revenir, et alors, ce sera la chute des prêtres. En attendant ce jour, il a rejoint l'Antalgis et a déjà libéré plusieurs de ses frères. En moins de 3 ans il s'est fait une petite réputation et a déjà du échapper à de multiples reprises à des fanatiques Teldrim.

Selon les rumeurs, le grand conseil Teldrim aurait décidé d'envoyer la Vakendar, un ordre d'assassins fanatiques, contre lui. Même sans ces assassins, Meisanne doit se méfier de tous Teldrims qui pourrait découvrir son secret et réagir selon ces preceptes religieux. 

Meisanne a également été contacté par un petit groupe, le Eshola. Ce groupe chercherait à recréer une prêtrise destiné aux esclave. Pour l'instant Meisanne n'a pas encore prit sa décision.

\end{document}
